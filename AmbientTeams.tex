\documentclass{seal_thesis}

\thesisType{Master Thesis}
\date{\today}
\title{AmbientTeams}
\subtitle{Staying socially connected in remote \\ knowledge work teams}
\author{Dario Bugmann}
\home{Aarau} % Geburtsort
\country{Switzerland}
\legi{15-708-852}
\prof{Prof. Dr. Thomas Fritz}
\assistent{Dr. André Meyer, Alexander Lill}
\email{dario.bugmann@uzh.ch}
\url{https://github.com/HASEL-UZH/PA.AmbientTeam}
\begindate{15.2.2021}
\enddate{22.7.2021}

\usepackage{subcaption}
\usepackage{float}
\usepackage{hyperref}
\usepackage{cleveref}
\usepackage{enumitem}
\usepackage{booktabs}
\usepackage{multirow}
\usepackage[english]{babel}
\usepackage{csquotes}
\usepackage{tabularx}
\renewcommand{\mkbegdispquote}[2]{\itshape}
\usepackage[toc,page]{appendix}
\usepackage[style=alphabetic]{biblatex}
\addbibresource{thesis.bib}


\usepackage{array}
\usepackage{arydshln}
\setlength\dashlinedash{0.2pt}
\setlength\dashlinegap{1.5pt}
\setlength\arrayrulewidth{0.3pt}

\renewcommand{\arraystretch}{1.1}

\begin{document}
\maketitle

\frontmatter

\begin{acknowledgements}
    I thank Prof. Dr. Thomas Fritz, Dr. André Meyer, and Alexander Lill for the opportunity to write this thesis, the time they took for it, and the constructive feedback they were always able to give me. I appreciate that they always listened to my thoughts and suggestions. I am very grateful for the beautiful trip to Vitznau where I could meet all the members of the HASE research group. A big thank you to all my study participants who took the time to help me with this project and gave such great feedback. I greatly appreciate my family and friends and am grateful for their support. Last but not least, a special thank you to Nora for all her love and support and the faith she always had in me.
\end{acknowledgements}

\begin{abstract}
    As remote work becomes more prevalent, informal and spontaneous conversations become less frequent because knowledge workers lack essential cues about their colleagues, such as their state of attention or current location.  The lack of such social interactions can lead to feelings of isolation at work. While existing approaches focus mainly on improving team awareness to ameliorate coordination and collaboration problems caused by remote work, fewer tools focus on fostering informal, spontaneous communication. To address this gap, our approach focuses on people, their moods and status, and opportunities for spontaneous interactions to create more social awareness. To this end, we developed AmbientTeams, an unobtrusive and informal tool that aims to reduce the perceived distance between distant colleagues. AmbientTeams strives to achieve this goal following a mood-based micro-blogging approach that allows individuals to share moods and status updates with the team and provides various opportunities for a subsequent response. In a preliminary evaluation, we tested our research prototype on a group of five knowledge workers who used the tool for one week. The results show that AmbientTeams facilitated getting to know each other by providing mood-based information. All in all, the encouraging results show that our novel approach, which allows knowledge workers to quickly and easily share moods with their team, has future potential as it enables and promotes a more informal, lightweight way of communicating.
\end{abstract}

\begin{zusammenfassung}
    Mit der zunehmenden Verbreitung von Fernarbeit werden informelle und spontane Gespräche seltener, da Wissensarbeiter keine wesentlichen Hinweise über ihre Kollegen erhalten, wie z. B. deren Aufmerksamkeitszustand oder aktuellen Standort.  Der Mangel an solchen sozialen Interaktionen kann zu Gefühlen der Isolation bei der Arbeit führen. Während sich bestehende Ansätze hauptsächlich auf die Verbesserung des Team-Bewusstseins konzentrieren, um Koordinations- und Kooperationsprobleme, die durch Fernarbeit verursacht werden, zu verbessern, konzentrieren sich weniger Tools auf die Förderung informeller, spontaner Kommunikation. Um diese Lücke zu schließen, konzentriert sich unser Ansatz auf Menschen, ihre Stimmungen und ihren Status sowie auf Gelegenheiten für spontane Interaktionen, um mehr soziales Bewusstsein zu schaffen. Zu diesem Zweck haben wir AmbientTeams entwickelt, ein unaufdringliches und informelles Tool, das darauf abzielt, die wahrgenommene Distanz zwischen entfernten Kollegen zu verringern. AmbientTeams strebt dieses Ziel an, indem es einen stimmungsbasierten Micro-Blogging-Ansatz verfolgt, der es Personen ermöglicht, Stimmungen und Status-Updates mit dem Team zu teilen und verschiedene Möglichkeiten für eine darauffolgende Reaktion bietet. In einer vorläufigen Evaluation haben wir unseren Forschungsprototyp an einer Gruppe von fünf Wissensarbeitern getestet, die das Tool eine Woche lang nutzten. Die Ergebnisse zeigen, dass AmbientTeams das gegenseitige Kennenlernen durch die Bereitstellung von stimmungsbasierten Informationen erleichtert hat. Alles in allem zeigen die ermutigenden Ergebnisse, dass unser neuartiger Ansatz, der es Wissensarbeitern ermöglicht, schnell und einfach Stimmungen mit ihrem Team zu teilen, Zukunftspotenzial hat, da er eine informellere, leichtere Art der Kommunikation ermöglicht und fördert.
\end{zusammenfassung}

\setcounter{tocdepth}{1}
\tableofcontents
\listoffigures
\listoftables
% \lstlistoflistings

\mainmatter
\chapter{Introduction}
\label{chapter:introduction}
Knowledge work has become increasingly distributed in recent years \autocite{herbsleb2001global}. This trend is caused by globalization, access to talent, cheaper labor,  and the increasing popularity of working from home \autocite{herbsleb2007global, ecoWorkingFromHome2021}. Reasons for this growing popularity include a more flexible schedule, increased work productivity, and spending less time and money on commuting \autocite{flores2019understanding, mulki2009set}. Additionally, the increased flexibility and autonomy allows employees to manage their family responsibilities better, leading to higher job satisfaction and employee retention \autocite{mulki2009set, gajendran2007good, madsen2011benefits}.

However, working remotely also brings challenges, namely that coordination and collaboration become much more difficult \autocite{herbsleb2007global}. One reason for this challenge is reduced team awareness, which is the understanding who is working on what, what they will do next, and how their actions might affect others \autocite{dourish1992awareness, herbsleb2007global, gutwin2004group}. %Workplace awareness defines awareness that results from the real-time combination of elements that workers keep in mind when collaborating \autocite{gutwin1996workspace}. Such elements can be people, what they are working on, what are are planning on working on next, which objects they are using, and more \autocite{gutwin1996workspace}.
This is because much of the implicit information (who is around, who can be disturbed, or who is currently working on what file) is no longer available when working remotely \autocite{gutwin2004importance}. Consequently, a large body of research has focused on improving these coordination and collaboration challenges by creating tools to increase awareness amongst team members (e.g., \autocite{biehl2007fastdash, jakobsen2009wipdash, cheng2003jazzing, deline2005easing}). These tools aim to improve collaboration efficiency by visualizing file navigation history, improving the understanding of other users' thought processes, visualizing co-workers' progress, or offering chat functionality where work-related conversations can be had, organized, and saved for later reference \autocite{biehl2007fastdash, jakobsen2009wipdash, cheng2003jazzing, deline2005easing}. 

However, previous work by \citeauthor{gutwin2004group} suggested that software developers can find all the information they need for their work even without such advanced awareness approaches \autocite{gutwin2004group}. Further, advances in commercial collaboration tools seem to develop rapidly (e.g., Google Workspace, JetBrains Space, GitHub, Microsoft Teams). For those reasons, our research focuses on more social challenges resulting from remote work, such as the scarcity of informal communication.

Due to the reduced awareness in remotely working teams, spontaneous, and often informal, communication is more difficult to initiate, and thus less prevalent in remote work \autocite{kraut1988patterns, sengupta2006research, herbsleb2007global, hinds2005understanding}. However, this is not desirable, because spontaneous or serendipitous communication, such as \enquote{corridor or watercooler talk}, accounts for about 85\% of all communication \autocite{kraut1990informal}, and can help to spread news faster among teams \autocite{herbsleb2000distance} or reduce coordination problems \autocite{herbsleb1999architectures} by gathering important background information that enables more effective teamwork \autocite{lanubile2007collaboration, herbsleb2001global}. 
%Besides less frequently engaging in spontaneous, informal communication, planned communication can suffer due to technical issues with video conferencing (e.g., low bandwidth), reducing remote team productivity \autocite{sengupta2006research}. 

The lack of such social interactions can lead to other interpersonal problems, such as difficulties in building trust, maintaining working relationships, or leading to feeling disconnected from the team \autocite{comella2020revisiting, olson2006bridging}. In extreme cases, a lack of social and emotional interactions can lead to workplace isolation \autocite{marshall2007workplace, gorlick2020productivity, mulki2009set}. This is critical since feeling disconnected from colleagues has been shown to decrease engagement in productive tasks \autocite{lostFocus2020}, while strong team cohesion has been shown to positively impact team effectiveness and productivity of a team \autocite{carlson2017virtual}. Research has thus also looked into ways of encouraging more social, spontaneous interactions within remotely working teams. One approach is virtual offices, which use virtual representations of an office where users can navigate around and interact with others and which have been developed both in research (e.g., \autocite{ lou2012presencescape}) and commercially (e.g., Branch\footnote{\url{https://branch.gg}}, Reslash\footnote{\url{https://reslash.co}}, Wonder\footnote{\url{https://wonder.me}}, or Gather\footnote{\url{https://gather.town}}). While \textcite{lou2012presencescape} noted an increase in informal communication through the use of their virtual worlds approach, we believe their 3D office visualization can be intrusive and therefore less suitable for everyday use in the workplace. 

Another, less intrusive concept aimed at promoting informal communication is micro-blogging in the workplace (e.g., \autocite{ebner2008microblogging, ehrlich2010microblogging, zhang2010case, dullemond2013fixing}). Micro-blogging is an informal form of communication where users can describe their current status, progress, or thoughts in short text messages and share them with other users \autocite{java2007we, dullemond2013fixing}. WeHomer, a micro-blogging tool introduced by \textcite{dullemond2013fixing}, was the first to extend a micro-blogging approach with mood sharing. Their motivation for sharing moods came from \textcite{garcia1999emotional}, who argued that being aware of the emotional state of your colleagues and acting accordingly leads to better collaborative work results.  However, this information is often lost in a remote setting because communication richness is drastically reduced because written text, which is often used among knowledge workers, has only limited ability to convey emotional data \autocite{hook2008interactional}. By studying WeHomer, \textcite{dullemond2013fixing} found an increase in team-connectedness and their participants had easy access to otherwise hard to obtain information. Despite their promising results, we note some limitations of their approach, namely that sharing moods was impossible without a status message, making it impossible to measure an isolated effect of sharing moods. In addition, the representation used for the moods was relatively inconspicuous by using text (e.g., \enquote{:-)}), leading us to believe that the effect of sharing moods was not very pronounced. Last but not least, responding to shared posts is only possible via commenting, which is visible to everyone else and therefore may not be ideal for more personal comments. Their findings and the fact that the COVID-19 pandemic has led to alarming numbers in employee well-being and mental health - 65.9\% of people report an increase in stress, and 44.4\% report a decrease in mental health \autocite{mswellbeing} - prompted us to develop a mood-based micro-blogging approach built on the foundation of WeHomer.

In our work, we extend WeHomer by addressing the identified limitations and studying the following key concepts:

\begin{enumerate}
    \item \textit{Focus on People and Micro-Blogging} \\
    Our approach focuses on people both visually and content-wise, and enables informal communication through mood-based micro-blogging.
    \item \textit{Spontaneous Interactions} \\
    Complementing micro-blogging, we believe that various opportunities for spontaneous interactions should be offered.
    \item \textit{Unobtrusive Design} \\ 
    Our approach focuses on moods by emphasizing them in a novel, unobtrusive user interface.
\end{enumerate}

Following these concepts, our goal is to increase social awareness and strengthen the sense of belonging to the team. This work aims to implement those concepts in a research prototype and evaluate their potential in a small preliminary evaluation. To do this, we investigate whether there is a general need for mood sharing in the workplace (RQ1), what users share with their team (RQ2), how users use our tool (RQ3), and what the broader implications of our approach are (RQ4). Thus, the research questions we sought to answer are:

\bigskip\noindent\textit{Information Sharing}

\smallskip\noindent\textit{RQ1}: Is there a need for sharing moods/states with team members, and what are the reasons?

\smallskip\noindent\textit{RQ2}: What are knowledge workers willing to share with their team?

\medskip\noindent\textit{Tool Usage and Workflows}

\smallskip\noindent\textit{RQ3}: How do knowledge workers use and interact with a mood-based micro-blogging tool? How do they integrate it into existing workflows?

\medskip\noindent\textit{Impacts of AmbientTeams}

\smallskip\noindent\textit{RQ4}: What are the effects of a mood-based micro-blogging tool?

\setlength{\leftskip}{0.5cm}
\smallskip\noindent\textit{RQ4.1}: Do mood and state sharing increase the awareness between team members, and how? What do they learn from each other?

\smallskip\noindent\textit{RQ4.2}: Does sharing moods and statuses affect the sharing user?

\smallskip\noindent\textit{RQ4.3}: Does a mood-based micro-blogging tool reduce the feeling of isolation in remote knowledge work teams?

% \smallskip\noindent\textit{RQ3.2}: Does it make users feel better to share information with their team?
% \smallskip\noindent\textit{RQ3.3}: Does it stress/relax users to see more about their team?

\setlength{\leftskip}{0pt}

\bigskip\noindent Following the concepts introduced above, we developed AmbientTeams, a desktop application that allows knowledge workers to add their most important team members and visualize them in a glanceable, transparent, and always-on-top window (see \autoref{fig:at_ambient_intro}). It differs from existing micro-blogging solutions in that it is more person- and mood-centric, and uses a novel approach to the user interface. Further, the content of textual information is de-emphasized as it is meant to complement the shared moods. These moods are visualized in mood-adapted avatars, which are the center of AmbientTeams. While \textcite{dullemond2013fixing} provides the ability to respond to shared posts with comments, AmbientTeams provides response options such as direct messaging and video conferencing to allow for spontaneous interactions. 

\begin{figure}[h]
    \centering
    \includegraphics[width=.5\linewidth]{./images/AT_ambient_intro.png}
    \caption{AmbientTeams: Screenshot of the Glanceable, Always-on-Top Window}
    \label{fig:at_ambient_intro}
\end{figure}

% Somewhat concerning to us is the seemingly little effort to improve the situation for remote workers suffering from the perception of workplace isolation, a feeling that is referred to as 

% Consequently, there is a decrease in communication frequency when physical distance between co-workers is increased {\autocite{herbsleb2003empirical} because much of the information normally present in a co-located setting is no longer available (e.g., whether a co-worker is in an available and an interruptable state (\autocite{})). 

% The current COVID-19 pandemic makes research in the field of remote work even more important because the majority of managers expect to have more flexible work from home policies post-pandemic, and employees would like to continue working from home at least partially \autocite{msworkindexconnection}, making the topic very much relevant also after the pandemic.

% While working from home has numerous benefits, it also comes with a range of challenges. On the benefits side, employers can realize savings in real estate costs, and the employee can benefit from more flexible work hours and spending less time and money commuting \autocite{mulki2009set}. However, shared challenges from working from home are that communication is reduced \autocite{kraut1988patterns} and suffers in quality \autocite{mulki2009set}. More specifically, informal communication drastically reduces when working from home \autocite{hinds2005understanding}. This reduction of informal communication can lead to difficulties building trust, maintaining work relationships, or not feeling attached to the team \autocite{comella2020revisiting, olson2006bridging}. Another consequence of remote work is the feeling of workplace isolation \autocite{mulki2009set, marshall2007workplace}. The feeling of isolation leads to not knowing whom to turn to in case of a problem or not feeling part of the company and is said to be caused by missing support from co-workers and opportunities for social and emotional interactions in a team \autocite{marshall2007workplace}. The pandemic further reinforces this influence leading to almost 60\% feeling less connected to their co-workers compared to before the pandemic \autocite{msworkindexconnection}. Since strong team cohesion has been shown to have a positive impact on the team's effectiveness and productivity \autocite{carlson2017virtual}, and the feeling of being disconnected from colleagues have been shown to impede engaging in productive tasks \autocite{lostFocus2020}, the connectedness with the team is of particular interest to us.

% A lack of awareness causes the challenges of working remotely: less information about co-workers is exchanged, e.g., no or fewer cues are available to identify team members' interruptibility or emotional states. This missing information makes it a lot harder to find opportune moments to initiate a conversation because it is often unknown whether a person might be in a deep focus state or whether a person might be more than happy to chat. Informal communication is further challenged because serendipity is missing when working remotely because people no longer randomly bump into each other at the water cooler or the coffee machine. Therefore, improving awareness in the workplace is the foundation of our approach.

% While there are several prior approaches to improve awareness within teams by showing the current coding tasks and work items that others are working on \autocite{biehl2007fastdash, jakobsen2009wipdash}, they do not focus on the person behind that work item. They thus do not put teams into the center of attention. To make this point stronger, recent research shows concerning numbers in regards to workers' well-being and mental health, stating that the pandemic has led to an increase in stress for 65.9\% of people and 44.4\% reported a decrease in mental health \autocite{qualtricksmental}. Therefore, our concept, AmbientTeams, follows a different approach by \enquote{putting people first}. It includes an ambient always-on-top overview of the core team members and their moods, status messages, and other states. In addition to such a micro-blogging approach, where team members can share information about their moods (and potential context), we aim to study further possibilities to foster and motivate serendipitous, informal exchanges with the team.

% In the next chapter, existing approaches and their underlying concepts are discussed before introducing our approach and its differences. The resulting prototype is then introduced in \autoref{chapter:prototype} and analyzed in the scope of a preliminary evaluation in \autoref{chapter:preliminary_evaluation}.


\bigskip\noindent To answer the research questions, we conducted a preliminary evaluation with five knowledge workers who used AmbientTeams for one week \textit{in-situ}. The participants confirmed the importance of staying aware of their co-workers' moods. Consequently, the mood-sharing functionality was the most popular feature among participants, primarily used without an attached status message. Regarding the broader effects of AmbientTeams, we found that it helped knowledge workers to 1) be more aware of each other's moods and availability status, 2) get to know each other better, 3) foster communication outside of AmbientTeams, and 4) spur self-reflection on one's moods. To summarize, the main contributions of this work include:

\begin{enumerate}
    \item the development of a mood-based micro-blogging approach with spontaneous interaction capabilities and
    \item the conduct of a preliminary evaluation leading to insights on awareness-raising and micro-blogging behaviors in remote teams and design considerations for such tools.
\end{enumerate}

%\begin{enumerate}
 %   \item Insights into mood and status sharing behaviors within knowledge work teams and the impact such sharing can have on personal relationships, workplace isolation, or collaboration
  %  \item Successful development of a glanceable, always-on-top status sharing window, and initial insights into the usability of such an approach
   % \item Provision and initial application of a study design that can be used for a broader study, and resulting suggestions for future features
%\end{enumerate}

The thesis starts with an overview of related work in \autoref{chapter:related_work} and continues with a discussion of the approach and its key concepts in \autoref{chapter:approach}. Subsequently, our research prototype and all its features are presented in \autoref{chapter:prototype}. The study design for the preliminary evaluation conducted can be found in \autoref{chapter:preliminary_evaluation} and the results in \autoref{chapter:results}. Last but not least, our findings are discussed and possible future directions of our approach are outlined in \autoref{chapter:discussion}.
\chapter{Related Work}
Remote work offers numerous benefits for both the employee and employer compared to traditional co-located work. Benefits on the employee side include a more flexible schedule, higher job productivity, and less time and money spent commuting \autocite{flores2019understanding, mulki2009set}. The increased flexibility and autonomy allows employees to more easily deal with their family responsibility and leads to higher levels of job satisfaction and higher employee retention \autocite{mulki2009set, gajendran2007good, madsen2011benefits}, both highly beneficial for the employer. The employer can further profit from savings in real estate costs and increased productivity \autocite{mulki2009set}. In addition to those general benefits, there is another popular reason for building distributed teams: the possibility to build teams with talents from all over the world \autocite{carmel1999global}.

However, remote work creates new challenges for the company and its employees. Therefore, it is not surprising that much research has been done in this area, most of which coming from Computer-Supported Collaborative Work (CSCW). The general goal of existing solutions is to support distributed teams in accomplishing work as effectively and efficiently as possible. While a lot of research goes into collaboration and coordination challenges in remote work, the goal of AmbientTeams is fostering social, informal interactions. As a result, we identified four main social challenges that are the result of working remotely. Together with existing solutions aiming at solving those problems, those four challenges are are discussed in this chapter.

% We categorize the main challenges of remote work into the following three distinct yet very related and interdependent categories: awareness, communication, and well-being. Keeping the challenges of working remotely in mind, we first take a closer look at those three concepts and then take a look at the existing approaches to solve those problems.

% \textbf{Interruptions and Work Fragmentation}\\
% Interruptions at the workplace have been shown to decrease productivity because when being interrupted, users require more time to complete tasks, make more errors, and experience more annoyance \autocite{bailey2006need}. Due to the lack of attentional states available when working remotely, interruptions cannot be timed as carefully \autocite{mark2005no}.

\section{Workplace Isolation}
\textcite{marshall2007workplace} define workplace isolation as the "[...] desire to be part of the network of colleagues who provide help and support in specific work-related needs. It represents employees' perceptions of availability of co-workers, peers, and supervisors for work-based social support." They further suggest a categorization into social isolation and organizational isolation. Oranzisational isolation stems from the perception that remote workers might feel "out of sight, out of mind" \autocite{bailey1999advantages}, while social isolation relates to the fact that remote workers miss informal chats, spontaneous discussions, and meetings around the water cooler \autocite{cooper2002telecommuting}. For those reasons, a closer look at communication and, more specifically, informal communication will be given in the following.

\section{Communication}
%Modularisation of source code is a prevalent design principle in the context of software engineering. However, software development is considered to be a highly complex task, meaning that even in very well-designed systems with high modularisation, some dependencies remain \autocite{cataldo2007coordination}. Many studies have shown that successful coordination during the process of software development leads to higher performance \autocite{kraut1995coordination}. Distributed teams have a disadvantage coordinating compared to traditional collocated teams \autocite{herbsleb2003empirical}.
Research in the field of software development states that co-workers are the most used source of information used by developers \autocite{ko2007information}, emphasizing the importance of team communication inside software development teams. When shifting from traditional, co-located work to remote work, studies find different results regarding the communication frequency. While \textcite{kraut1988patterns, allen1984managing} find a decrease in communication, \textcite{mulki2009set} find increased communication in a remote setting. A possible reason for more communication includes the need for remote workers to over-communicate their availability status to their co-workers \autocite{koehne2012remote}. Reasons for communication reduction could be the active effort to bring back ad-hoc meetings \autocite{miller2021your}, or the lack of the required awareness to initiate a conversation. Regardless of communication frequency, working remotely and thus using software to communicate leads to having more misunderstandings due to missing cues and thus reduces communication effectiveness \autocite{mulki2009set}. Other researchers argue that the reason for those misunderstandings is the fact text-based communication (which is often used in software development) has very limited capacity, and thus a lot of socio-emotional information (non-verbal cues) is missing \autocite{hassib2017heartchat}. This most likely is a reason why face-to-face communication is still very important for many developers \autocite{storey2016social} and a lack thereof, which is caused by working remotely, can lead to workplace isolation making it harder to develop personal relationships and build trust \autocite{mulki2009set}. \textcite{gajendran2007good} state that working from home with high-intensity (more than 2.5 days a week) harmed relationships with co-workers, something that is enforced because of the Covid-19 pandemic. Since informal communication helps developing work relationships \autocite{comella2020revisiting, olson2006bridging}, it is of special importance in distributed teams.


% This is likely more pronounced because of the global pandemic, which forced teams to switch from collocated to remote work, making communication as accessible as possible should be desirable regardless.


\subsection{Informal Communication}
\textcite{kraut1990informal} define informal communication as "[...] communication that is spontaneous, interactive and rich". Differences to formal communication include lack of planning and the fact that the content of the communication is unknown in advance. \textcite{kraut1990informal} further state that over 85\% of all conversations are informal and informal communication happens more often if there is a short physical distance between parties. Similarly, \textcite{hinds2005understanding} find that members of distributed teams engage less in informal conversations. This reduction of informal communication is unfortunate since informal communication is crucial for achieving high productivity and social goals \autocite{kraut1990informal} such developing work relationships \autocite{comella2020revisiting, olson2006bridging}. More concretely, in the field of software development, informal communication plays a critical role due to the fast speed at which informal communication distributes knowledge across a team or company \autocite{french1998study, mockus2001challenges}. Also, informal communication can increase awareness, enabling developers to work efficiently \autocite{herbsleb2001global}. In the ever-changing field of agile software development, this is particularly useful because requirements can change, and formal communication channels cannot spread the news as fast. Besides, informal communication is essential for conflict identification and handling \autocite{hinds2005understanding}. The fact that teams with a high degree of social interactions often have better team cohesion \autocite{staehle2014management}, and informal communication is normally much more frequent than formal forms of communication \autocite{kraut1990informal}, further pronounces the importance of informal communication.

Because of those benefits, it is no surprise that numerous approaches are fostering informal communication inside distributed teams. One of the earliest proposed solutions for promoting informal communication in distributed teams was VideoWindow \autocite{fish1990videowindow}. Despite being an early solution, the authors already identified two essential requirements such a system must offer: low personal cost and the need for a visual channel. Suppose the prices for initiating conversations are too high. In that case, the system will not be helpful because the tool will not be used. The visual channel also plays a vital role by recognizing the presence of other people, indicating whether a conversation can be initiated. \textcite{sasaki1999video} developed a hallway system that was able to raise awareness and helped to indicate that one might have a question but failed to promote casual interactions. In comparison, \textcite{lou2012presencescape} manages to provide awareness information that is relevant to engage in everyday conversations and a low-effort mechanism to initiate such informal discussions. It does so by providing social cues which help understand the availability of others and thus creating a context for subsequent communication.

As a consequence of the global pandemic, many commercial tools have been published recently. Branch\footnote{\url{https://branch.gg}}, Reslash\footnote{\url{https://reslash.co}}, Wonder\footnote{\url{https://wonder.me}}, or Gather\footnote{\url{https://gather.town}} also follow the goal of increasing spontaneous, informal communication by creating virtual offices where users can move around with avatars and interact with others. Tandem\footnote{\url{https://tandem.chat/}} is another tool with a focus on collaboration and takes a less playful approach by being more similar to traditional communication apps user interfaces.

Another form of communication that has been studied extensively is the concept of microblogging. Studies have shown that microblogging is a form of informal communication \autocite{ehrlich2010microblogging} that is "[...] like a virtual coffee machine as a meeting place" \autocite{ebner2008microblogging}. Further, many existing microblogging approaches have found that microblogging results in people feeling more connected \autocite{ehrlich2010microblogging, zhang2010case}. Likewise, their study participants found microblogging very helpful because it allowed them to stay aware of what their team members are doing \autocite{zhang2010case}. In addition to purely share text-based contents, which is the standard in microblogging, \textcite{dullemond2013fixing} developed a microblogging system that allows the users to attach a mood to each message which helped the teams feeling more connected. What they did not measure, however, is the isolated effect of mood sharing.

Due to the value of providing additional awareness and sharing moods in the workplace, the following two sections focus on those two concepts.

\section{Awareness}
A reason for coordination and communication challenges in a remote work environment is the lack of awareness, so it is of great interest to increase awareness in distributed teams. \textcite{gutwin1996workspace} defines group awareness as a combination of:

\begin{itemize}[itemsep=0ex, parsep=0ex, leftmargin=*]
      \item \textbf{Informal Awareness} \\
            The general sense of the presence, availability, and activities of others. It is the "glue that facilitates casual interactions" \autocite{gutwin1996workspace}.
      \item \textbf{Group-Structural Awareness} \\
            "Group-structural awareness involves the knowledge about people's roles and responsibilities, their positions on an issue, their status, and group processes" \autocite{gutwin1996workspace}.
      \item \textbf{Social Awareness} \\
            "Social awareness is the information that a person maintains about others in a social or conversational context "\autocite{gutwin1996workspace}. It includes things as the attention state of the other person, their emotions, or the level of interest \autocite{gutwin1996workspace}, or whether a person can be disturbed \autocite{gutwin1995support}.
      \item \textbf{Workplace Awareness} \\
            Defines the awareness that results from the 'real-time' combination of elements workers keep track of when working together. Such elements could be people, actions, objects, and many more \autocite{gutwin1995support}.
\end{itemize}

It is important to note that those four awareness types are not excluding but rather overlapping with each other. Put differently, informal, social, and group-structural awareness are all part of workplace awareness. In the case of software developers, for instance, a study shows that developers checked the availability status of their co-workers almost as many times as their compiler output \autocite{ko2007information}. This indicates the importance of informal awareness. Providing group-structural is essential because of difficulties when trying to find experts in a distributed team \autocite{herbsleb2003empirical}. Social awareness is highly relevant due to the high communication needs of software developers \autocite{perry1994people}. Additionally, with less face-to-face communication and more computer-mediated communication, it is consequently more difficult to transfer emotional information \autocite{rivera1996effects}.

Despite the seemingly precise categorization of awareness above, the literature does not agree on one common categorization. Alternative terms such as general awareness, peripheral awareness, co-existent and cooperation awareness, and objective self-awareness are used to describe and categorize awareness. Due to the popularity and granularity of the above definition, it is the definition of choice for this work.

To address the problem of missing awareness when working remotely, a wealth of research developed approaches to increase awareness in distributed teams. Popular tools made explicitly for software development teams focus on providing awareness by on work items, developers' activities (e.g., which files they have opened or recently changed) and thus put the code base and tasks in the foreground of coordination \autocite{biehl2007fastdash, jakobsen2009wipdash, eick1992seesoft, deline2005easing}. \textcite{cheng2003jazzing} introduces JazzBand, an IDE plugin visualizing the team members to increase peripheral awareness enhanced with status messages and chat functionality facilitating coordination.

While the majority of the above-mentioned awareness tools require user interactions to be helpful, there have also been attempts for creating ambient approaches to raise awareness in the work environment \autocite{morrison2020facilitating, otjacques2006ambient, downs2012ambient, alavi2012ambient, rocker2004using}. \textcite{downs2012ambient} define ambient devices as devices that "[...] present dynamic information in an at-a-glance manner and have low attentional requirements".

\section{Well-being: Emotions, Moods, and Sentiments}
A common finding in research regarding remote work is that employees work longer hours, experience more stress, and have difficulties with mental health \autocite{mswellbeing, mulki2009set, qualtricksmental}. A recent study in the context of the global Covid-19 pandemic lists the negative impacts from working from home, such as increased burnout, lack of separation between work and life, and feeling disconnected from co-workers \autocite{mswellbeing}. A Psychological study highlights that the mental health of remote workers should be considered and is very important to be communicated and talked about \autocite{grant2013exploration}. Yet, emotions can get lost or misunderstood inside text messages due to the lack of cues in text-based communication \autocite{hook2008interactional}. For this reason, \textcite{kuwabara2002connectedness} highlights the need for connectedness-oriented communication because it is critical for developing social relationships and harder to do over distance. \textcite{mcduff2012affectaura} further state the usefulness of being able to assess one's emotional state (e.g., when considering mental health issues). Their approach, AffectAura, is developed using different kinds of sensors to predict emotions and provide an overview of them in a diary-like fashion with the purpose of self-reflection \autocite{dullemond2013fixing}. \textcite{guzman2013towards} emphasizes the importance of emotion in software development, however focusing on the emotional state towards a project, not of individuals. MobiMood is a mobile application focusing on individuals by letting them share their moods, but not targetting a work environment \autocite{church2010study}. \textcite{saari2008mobile} developed another mobile application with mood sharing features aimed at knowledge workers. However, while the researchers developed the prototype, their approach's usability and use cases were not studied.


% Apart from the importance of well-being for personal health, there are also more direct impacts on work. For instance, different moods of programmers have been found to have an effect on debugging performance \autocite{khan2011moods}. 

To communicate one's well-being, different types of affective responses exist that can be useful for sharing with the team, namely emotions, moods, and sentiments. Emotions are typical reactions to events and therefore have a definite cause and are typically short-lived. Emotions differ from moods in that moods are longer in duration, have no clear target, and are less intense \autocite{frijda1994varieties, brave2007emotion}. Sentiments can be described as states associated with objects rather than individuals and therefore are relatively permanent \autocite{brave2007emotion}.

When it comes to measuring well-being, the literature does not reach a consensus of one best measurement. However, the valence-arousal dimensional model is most commonly referred to as the better model \autocite{russell1980circumplex, mauss2009measures}. It is a two-dimensional model where the arousal dimension contrasts states of pleasure with states of displeasure, and the arousal dimension contrasts states of low arousal with states of high arousal \autocite{mauss2009measures}. Results of this model can then be used to map onto a discrete set of basic emotions such as surprise, fear, disgust, anger, happiness, or sadness \autocite{brave2007emotion}.
\chapter{Approach}

 "Developers checked on coworker availability almost as many times as they looked at output from the compiler or debugger" \cite{ko2007information}. -> important to make it as easy as possible -> ambient display
 
 ambient display -> Presence awareness
\chapter{Research Prototype}
\label{chapter:prototype}
The above outlined key concepts were then developed into the key features of our research prototype, \textit{AmbientTeams}. Before stepping into the core features employed in AmbientTeams and aligning them to the above-mentioned key concepts (see \autoref{chapter:approach}), a brief introduction into the more technical aspects and a general overview of the application are given.

\section{Architecture}
AmbientTeams is a cross-platform desktop application based on Electron\footnote{\url{https://www.electronjs.org/}}. To facilitate the implementation of the interactive user interface in AmbientTeams, VueJS\footnote{\url{https://vuejs.org/}} is used as the JavaScript framework for the front-end. To maintain JavaScript as a common language for the front-end and back-end, NodeJS\footnote{\url{https://nodejs.org/}} is used on the server-side. The server provides both a REST API for basic CRUD functionality for users and teams and a WebSocket endpoint since much of the data required for AmbientTeams comes from the server in real-time.

\section{Teams and \enquote{Favorites}}
There are two types of teams in AmbientTeams; regular teams are stored on the server and require a unique identifier to join, similar to a straightforward invite-based approach often seen in practice. For scenarios where a user is part of multiple such teams, team members from different teams can be linked to a \enquote{favorites} team. These favorite teams only exist on the local machines of the users. In general, there is no visual difference between the two types within AmbientTeams, except that 1) there is no always-on breakroom for Favorite Teams, and 2) team members of a Favorite Team combine all status and direct messages when a Favorite Team member is part of multiple regular teams. This is different from regular teams, where status messages and direct messages are limited to that one team.

\section{Avatars}
At the core of our approach are the avatar representations of the users. While we could have opted for traditional profile pictures that allow users to upload an actual photograph, we decided to use the abstract form due to privacy reasons, allowing relatively simple mood manipulation on such avatars. Also, using an avatar library gives the user interface a more clean, uniform look, which is why we make use of getavataaars\footnote{\url{https://getavataaars.com}} to create and manipulate avatars. Users are asked to create their own avatar during the sign-up process and have the possibility to change the appearance later on. To represent the currently selected mood of each user, AmbientTeams automatically adjusts the eyes, eyebrows, and mouth types supported by the getavataaars' Application Programming Interface (API) to best possibly represent the selected emoticon.

\section{Windows}
AmbientTeams consists of two main windows; the team overview and the ambient window.

\subsection{Team Overview Window}
The team overview window is responsible for maintaining a connection to the server, authenticating, login functionality, settings. Additionally, once users have authenticated inside the team overview window, they are redirected to the team overview view where all teams and team members are visible (see \autoref{fig:at_overview}). By clicking on the edit icon next to the team name, the user can select team members from each team that will then be displayed on the other main window, the ambient window. This is demonstrated in \autoref{fig:at_overview}, where the user is selecting the team members to be displayed on the ambient window. In summary, apart from authentication purposes and initial application setup, the team overview window is primarily intended for people who are part of multiple teams and want to get a quick overview of all the different teams they belong to.

\begin{figure}[h]
    \centering
    \includegraphics[width=.8\linewidth]{./images/AT_overview.png}
    \caption{Team overview window }
    \label{fig:at_overview}
\end{figure}

\subsection{Ambient, Glanceable Window}
The Ambient window is always on top of other windows (see \autoref{fig:at_ambient_on_top}), which on the one hand, makes it easy to stay informed about moods and other statuses of your team members, but on the other hand, can also cause interruptions and distractions. We used a transparent borderless window to keep the ambient overlay as ambient and unobtrusive as possible. However, if the window is still distracting, it can be easily minimized or closed altogether using the menu that can be accessed by clicking the minimize icon (see \autoref{fig:at_hover}). Opening the team overview window can be achieved by clicking on the three dots in the menu, which will display a small drop-down menu with the option to open the team overview window. Also, in this menu, the ambient window can be enlarged or reduced to fit different screen resolutions and personal preferences.

\begin{figure}[h]
    \centering
    \includegraphics[width=.8\linewidth]{./images/AT_ambient_on_top.png}
    \caption{Always-on-top ambient window while working on another task }
    \label{fig:at_ambient_on_top}
\end{figure}

Further, certain functionality is only visible when the user is hovering over this window. When hovering over the ambient window, the user can select the team they want to show and sees the names of the individual team members, as shown in \autoref{fig:at_hover}.

\begin{figure}[h]
    \centering
    \begin{subfigure}{.5\textwidth}
        \centering
        \includegraphics[width=.8\linewidth]{./images/AT_no_hover.png}
        \caption{No hover }
        \label{fig:at_no_hover}
    \end{subfigure}%
    \begin{subfigure}{.5\textwidth}
        \centering
        \includegraphics[width=.8\linewidth]{./images/AT_hover.png}
        \caption{Hover }
        \label{fig:at_hover}
    \end{subfigure}
    \caption{Ambient window}
\end{figure}

\section{Availability Status}
AmbientTeams wants to keep the number of interruptions to a minimum, which is why there is the \enquote{Focused} availability state (see \autoref{fig:call}) that exists in addition to the three others (\enquote{Available}, \enquote{Offline}, and \enquote{Happy to Interact}). Users in this focused state cannot be called. Further, they don't see any direct messages or incoming nudges until they leave the focused state. In addition, focused users cannot directly interact with other team members, avoiding potential self-distraction. The availability state \enquote{Happy to Interact} was included to address the lack of serendipity in remote work. When selected by at least two team members, an automatic matchmaker runs every minute and randomly pairs two people, who are then routed to a video call.

\section{Sharing Moods and Status Messages}
The user can open the sharing window from both the team overview and the ambient window, and the system tray menu. All of those actions will open the sharing window as shown in \autoref{fig:sharing_manual}, where on the left, a preview of the current avatar and the selection of available moods are listed. There are nine available moods, visualized using popular emoticons available through OpenMoji\footnote{\url{https://openmoji.org}}, an open-source emoji project. The first four of the available emoticons are more optimistic, the fifth is a neutral face, and the last four are emoticons representing rather negative emotional states. The selection of the emoticons started with six basic emotions: surprise, fear, disgust, anger, happiness, and sadness \autocite{an2017two}. This list was expanded over time to better suit the work environment (TODO: Find references to support some of the moods/emotions that we added) by adding a neutral and tired emoticon and two more positive emotions (loving hearts and grinning) to make the selection more balanced. Due to limitations with the avatar API, we could not render \enquote{fear} well enough, which led us to remove it. On the right, the user can enter additional context in a simple, standard textbox. The contents of this textbox are, if available, pre-populated with the current status message for the currently selected team. Additionally, the text is highlighted when the window is created, facilitating overwriting the current status without using the mouse to select the text manually. Status messages' length is limited to 140 characters, motivated by the initial limit of Twitter \autocite{dullemond2013fixing}. Below the textbox, the user can find a button to share the status message with either all teams or a single team.

As a reminder for the user to share their moods and potential additional context with team members, the sharing window also appears automatically at pre-defined times. The location we chose for this popup is the lower right corner of the user's primary monitor to minimize the potential for distraction. Overall, the window has the same functionality but includes two additional buttons to defer the prompt for either 5 minutes or 1 hour (see \autoref{fig:sharing_auto}). The scheduled sharing window is displayed at three pre-defined times throughout the day, namely at 9:00, 13:00, and 16:00 local time.

\begin{figure}[h]
    \centering
    \begin{subfigure}{.5\textwidth}
        \centering
        \includegraphics[width=.8\linewidth]{./images/sharing_manual.png}
        \caption{Manually opened }
        \label{fig:sharing_manual}
    \end{subfigure}%
    \begin{subfigure}{.5\textwidth}
        \centering
        \includegraphics[width=.8\linewidth]{./images/sharing_auto.png}
        \caption{Scheduled }
        \label{fig:sharing_auto}
    \end{subfigure}
    \caption{Sharing window}
\end{figure}

To ensure that the information shared within AmbientTeams is up-to-date, a few measures have been taken. The first is purely visual: avatars are increasingly hidden the longer there has been no current activity. Such activities include status and mood sharing, direct messaging, and nudging. This automatic hiding should motivate users to interact with such hidden team members and easily spot updates from colleagues. Another measure we have taken to avoid showing users outdated content is automatically resetting status updates and moods at midnight.

Since the goal of AmbientTeams is to encourage informal communication, there is no chat history or other history built into the application. With this impermanence, we want to promote more casual and less formal communication and hope to avoid AmbientTeams becoming just another tool to keep track of work.

\section{Ever-Running Breakroom}
As mentioned before, our goal was to create ever-running breakrooms as effortlessly as possible. \autoref{fig:breakroom_initiator} shows the state of the ambient window when the user has clicked on the coffee icon. After the user clicks on this coffee icon, the other team members will see an indication that there is a breakroom in progress (see \autoref{fig:breakroom_indicator}). However, to avoid unnecessarily creating a breakroom and potentially interrupting the initiating user, the breakroom is not created until another user clicks on the coffee icon.

\begin{figure}[h]
    \centering
    \begin{subfigure}{.5\textwidth}
        \centering
        \includegraphics[width=.8\linewidth]{./images/breakroom_initiator.png}
        \caption{Initiating a breakroom creation }
        \label{fig:breakroom_initiator}
    \end{subfigure}%
    \begin{subfigure}{.5\textwidth}
        \centering
        \includegraphics[width=.8\linewidth]{./images/breakroom_indicator.png}
        \caption{Joining a breakroom }
        \label{fig:breakroom_indicator}
    \end{subfigure}
    \caption{Breakroom creation}
\end{figure}

Once at least two team members have clicked the breakroom icon, a breakroom is created in the back-end with twilio\footnote{\url{https://www.twilio.com}}, and they are redirected to the breakroom view (see \autoref{fig:breakroom}). At any point, other team members can join and leave the breakroom, and it will remain active as long as at least one team member is present. We want to avoid users forgetting the time and staying too long in the breakroom. For this purpose, a 15-minute timer is started as soon as one enters the breakroom. When this timer reaches its end, the user automatically leaves the breakroom.

\begin{figure}[h]
    \centering
    \includegraphics[width=.8\linewidth]{./images/breakroom.png}
    \caption{Breakroom, pictures are artificially created with \url{https://generated.photos/}}
    \label{fig:breakroom}
\end{figure}

\section{Direct Interactions}
In addition to broadcasting sentiments and status messages, there is also the ability to interact directly with an individual team member. Hovering over individual team members brings up an overlay that offers three different interaction options, namely 1) direct messaging, 2) nudging, and 3) direct calling (see \autoref{fig:interactions}).

\begin{figure}[h]
    \centering
    \includegraphics[width=.4\linewidth]{./images/interactions.png}
    \caption{1:1 interaction overlay }
    \label{fig:interactions}
\end{figure}

Direct messaging is very similar to status message sharing but without mood sharing and team selection options. After clicking the message icon, the message window (\autoref{fig:messaging_window}) is displayed at the user's current mouse position to minimize the distance needed to interact with the window's contents. As in the status sharing window, there is a character limit of 140 characters.

\begin{figure}[h]
    \centering
    \includegraphics[width=.4\linewidth]{./images/messaging_window.png}
    \caption{Messaging window }
    \label{fig:messaging_window}
\end{figure}

In \autoref{fig:interaction_results} all three interaction options are visualized. Direct messages are distinguished from status messages by the message icon located to the left of the actual message. Nuding uses a hand icon pointing to the team member in question. For a video call, the video stream overlays the team member's avatar, and the availability status of both participants is automatically set to \enquote{Focused}. Users can hover over their avatar if they want to mute or pause the video stream. To end a call, you need to hover over the corresponding team member and click the hang-up icon.

\begin{figure}[h]
    \centering
    \begin{subfigure}{.3\textwidth}
        \centering
        \includegraphics[width=.9\linewidth]{./images/DM.png}
        \caption{Direct message }
        \label{fig:dm}
    \end{subfigure}%
    \begin{subfigure}{.3\textwidth}
        \centering
        \includegraphics[width=.9\linewidth]{./images/nudging.png}
        \caption{Nudging }
        \label{fig:nudging}
    \end{subfigure}
    \begin{subfigure}{.3\textwidth}
        \centering
        \includegraphics[width=.9\linewidth]{./images/call.png}
        \caption{Ongoing video call }
        \label{fig:call}
    \end{subfigure}
    \caption{1:1 interactions}
    \label{fig:interaction_results}
\end{figure}
\chapter{Preliminary Evaluation}
To evaluate the above mentioned research questions, a small pilot study was conducted. Optimising and improving our approach with the help of feedback from a small team consisting of knowledge workers is the main goal of this master thesis. Further, by collecting additional data hope to get valuable first insights into the following areas:

\begin{enumerate}
    \item Common workflows and patterns \\
          \textit{RQ1:} How do knowledge workers use and interact with AmbientTeams? How do they integrate it into existing workflows?
    \item Mood sharing behavior \\
          \textit{RQ2:} Do knowledge workers want to share their moods with their remote team members? If so, what causes knowledge workers to share their moods with their team?
    \item Feeling of isolation \\
          \textit{RQ3:} Can a people-focused team mood and status sharing approach decrease the feeling of isolation in remote knowledge worker teams?
    \item Information consumption \\
          \textit{RQ4:} What do knowledge workers learn from information shared by their team members? What kind of information is the most valuable?
\end{enumerate}

The data used to answer the above research questions came from three different sources. Given the relatively few participants, it was important to have both quantitative and qualitative data. After having a look at the study procedure in the next chapter, each of the three data sources and their relevance for the research questions are elaborated.

To find participants, a recruitement flyer (see .) The study was conducted with two knowledge worker teams, each team fulfilling the following paritication requirements:

\begin{enumerate}
    \item At least three team members
    \item Three or more common working days a week
    \item Spending the majority of their work day on the computer
    \item Having all the required rights to install AmbientTeams on their work computer
    \item Willingness to use AmbientTeams during at least 3 full days of work (approximately 0800 - 1700)
    \item Using macOS or Microsoft Windows
    \item An active internet connection
\end{enumerate}

\section{Procedure}

\begin{figure}[h]
    \centering
    \includegraphics[width=.8\linewidth]{./images/Study_Timeline.png}
    \caption{Study timeline}
    \label{fig:study_timeline}
\end{figure}

\begin{enumerate}
    \item Initial meeting: Installation, registration, team creation, give each team member a participant ID (see \autoref{section:initial_meeting})
    \item Pre-study questionnaire (see \autoref{section:questionnaires})
    \item Evaluation phase (see \autoref{section:evaluation})
    \item Post-study interview (see \autoref{section:interview}) with data collection (see \autoref{section:data_collection})
    \item  Post-study questionnaire (see \autoref{section:questionnaires})
\end{enumerate}

\section{Initial meeting}
\label{section:initial_meeting}
Installation

\section{Pre- and post-study Questionnaires}
\label{section:questionnaires}
The pre and post study questionnaires are used to asses the feeling of workplace isolation before and after the evaluation period. The questions are taken from ....

\section{Evaluation Phase}
\label{section:evaluation}
How long? \\
During the study notes from notion

\section{Interview}
\label{section:interview}
Interview questions and their relevance for the research questions

\section{Data Collection}
\label{section:data_collection}
\autoref{table:data_collection} shows an overview of all data collected and for which research question they are relevant.

\begin{table}[h]
    \centering
    \begin{tabular}{ |c|c|c| }
        \hline
        Data collected & Storage & Relevant for \\
        \hline
        cell4          & Local   & cell5        \\
        cell7          & Local   & cell8        \\
        \hline
    \end{tabular}
    \caption{The data collected during the preliminary evluation and its relevance for the RQs}
    \label{table:data_collection}
\end{table}
\include{chapters/results}
\chapter{Discussion}
\label{chapter:discussion}
This chapter discusses the key findings from our preliminary evaluation and elaborates on possible future directions that our approach, AmbientTeams, could take.

\section{Ambient Window}
\label{section:ambient_window_discussion}
In contrast to our expectations, the ambient window was not necessarily the primarily used window when interacting with AmbientTeams. While the high usability scores that resulted from the SUS questionnaire show that AmbientTeams is perceived as intuitive and easy to use, some participants mentioned that they were annoyed of the ambient window, as it always was in the foreground. The inconvenience and difficulty of positioning the ambient window is a fairly crucial issue that may require further development of AmbientTeams; the case described by P2 (closing AmbientTeams due to established habits) could be resolved in a future study by not allowing the closing of the ambient window. Further, we believe that another reason for the little usage of the ambient window could be that its somewhat novel and unknown approach compared to more traditional applications. In fact, P5 stated exactly that: \textit{\enquote{I think it [reason for not using the ambient window] was some insecurity on my part}}. Therefore, a more extended period of usage and getting used to such a window might be required.

All things considered, and since there were no complaints that the content displayed in the ambient window was distracting in any way, we are optimistic about the ambient window and its general appearance. The suggested improvements for better usability are all feasible and could be implemented with reasonable effort. 

\section{(Mood) Awareness and Informal Interactions}
Our participants confirmed that they experience a lack of mood awareness and social interactions when working remotely, reinforcing our motivation for developing AmbientTeams. As the in-house breakroom is well attended and gives new employees the chance to meet each other, it suggests that such a breakroom concept is generally perceived as essential and partially explains why the breakroom integrated into AmbientTeams might not have been used. 

Since the mood sharing functionality was used more extensively, AmbientTeams managed to increase awareness, primarily mood awareness. Like \textcite{church2010study}, who stated that being aware of one's moods could act as a springboard for communication, we obtain similar results. While we have not been able to identify lots of informal communication within AmbientTeams, we see an increase in other communication applications such as Microsoft Teams or Skype directly after AmbientTeams was used. While this does not necessarily guarantee that a conversation took place at all or that it was informal, it seems that interactions immediately following AmbientTeams seem to take place on more informal platforms such as Skype, which still leaves us feeling optimistic. 

In contrast to the beliefs of \textcite{garcia1999emotional} that emotions would improve the effectiveness of collaboration, we cannot make such statements about our participants. However, our participants mentioned that they would be likely open to help should they notice that their colleagues shared a negative mood. 

\section{Sharing Behavior}
Likewise, our mood-based micro-blogging approach seems not to have impacted any work-related awareness since exclusively non-work-related content was shared. This is in contrast to the study performed on WeHomer \autocite{dullemond2013fixing}, which found that the content of the shared status messages was often personal or non-work-related but also included work-related information. Reasons for this discrepancy could be that our finding is 1) highly company-specific or 2) not comparable due to lack of data points. Despite these concerns, however, we believe that one possible explanation is that our mood-centered visualization approach, coupled with the transient nature of the displayed content, primarily supported non-work-related exchanges: if there is no chat history, the contents inevitably are less likely to be formal because the chat no longer serves as a \enquote{knowledge archive}.

We believe that this is also the reason why moods were shared far more frequently than status messages. On top of that, we hypothesize that the visual representation of AmbientTeams stirred the participants unconsciously towards sharing more moods and non-work-related information. Regardless of the reasons for such behavior, we see this as a win as it positions AmbientTeams in a somewhat niche sector of communication tools.

To our surprise, the reasons behind deciding which moods should be shared or which not, particularly negative moods, were often brought up in the final interviews. While \textcite{dullemond2013fixing} found similar results in the sense that positive moods were shared more frequently, we were able to gather first insights into the reasons why people might be hesitant to share negative moods. 

A possible future update to the sharing mechanism currently in use could include allowing content to be shared with only a subset of the team. This could be beneficial as not all moods, and status updates are intended for the entire team (P3). 

\section{More Extensive Study}
In order to make more meaningful statements about AmbientTeams, a more extensive study is an absolute requirement, especially considering that certain functionality was not used during our preliminary evaluation. Such a study would involve more teams, ideally teams that differ in various aspects such as industry, age of participants, and corporate culture. Based on the interviews, we have reason to believe that \textit{age} (P5) and \textit{company culture} (P2) may be predictors of willingness to share moods in the workplace. Also, the study design worked well in the small setting of this thesis. No significant problems were reported concerning the tool. For those reasons, we see no problem with conducting the same study - except for some minor adjustments to the interview questions and questionnaires, and potentially the ambient window as explained in \autoref{section:ambient_window_discussion} - in a larger setting. However, inputs gathered from the small study also outline some potential future directions and further developments of the tool, AmbientTeams, which could be realized before continuing with a more extensive examination. We discuss those possible updates in the following sections.

% - P2: "sometimes you have this barrier to share the mood, so you have to establish this culture in the team, that it becomes a normal thing that everyone shares their angry or sad mood."

\section{Focus on Asynchronous Communication}
The lack of used synchronous communication features (video/audio calls) leads us to believe that the more realistic and promising approach would be to focus exclusively on the parts of asynchronous communication that are not yet integrated into existing communication tools (e.g., Slack, Microsoft Teams, Zoom.us). This change would mean that the main functionality of AmbientTeams would be limited to sharing moods and status messages. The functionality to nudge or directly notify other team members could also be retained as a potential communication trigger. Given our findings and the fact that most companies have established a communication culture using a software solution with advanced collaboration features, we believe that AmbientTeams should not compete with such tools but rather focus on what is different from them: mood sharing and informal status sharing. However, these are findings that we obtained from a small group of participants, all part of one company, and therefore cannot be generalized. Regardless, simplifying AmbientTeams would also have the benefit of making it easier for study participants to learn to use, as there are fewer features to learn and discover.

\section{Better Integration With Established Tools}
\label{section:better_integration_with_established_tools}
Participants P2, P3, and P5 indicated that they would prefer to have only one application for their team communications. P2 argued that a single tool would increase the likelihood and time that they would use AmbientTeams. We see two ways we could improve the use and user experience of AmbientTeams in the long term: 1) two-way synchronization of data between existing tools and AmbientTeams, or 2) complete relocation of AmbientTeams functionality to existing tools. The former means that AmbientTeams would remain a standalone desktop application and continue to benefit from the freedom it provides. It would use application programming interfaces (APIs) to push and pull updates to and from existing communication tools. Potential information that could be shared includes availability status and status messages. To maintain \enquote{one click on the avatar to start a call}, it could also leverage the more mature video conferencing capabilities of the existing solution for more seamless interaction between AmbientTeams, making it less like \enquote{just another tool} and more of a potential facilitator for using existing tools. The second measure would go in a completely different direction, essentially moving all of AmbientTeams' existing functionality into existing ecosystems like Microsoft Teams or Slack. While this would satisfy our participants' desire for a universal communication tool, we would also lose a lot of flexibility. The ambient window would have to go, and it's not yet clear how much of the functionality we could adopt. More research would need to be done on the capabilities of these established communication platforms before ultimately deciding on the better approach.

\section{Self-Reflection}
Feedback from P3, P4, and P5 told us that there is a genuine interest and potential benefit in reflecting on one's mood. Therefore, one possibility would be for AmbientTeams to move more towards self-reflection in the future. This could be achieved through various new or slightly modified existing features. For example, when selecting a mood, the user could be asked via emoticons if they would like to share the selected mood or update the local mood. A dashboard could then provide the user with various visualizations to reflect on moods, similar to AffectAura \autocite{mcduff2012affectaura}. Furthermore, P3 mentioned that linking tasks to moods would be of high interest. Again, this is similar to AffectAura's functionality of linking emotions to artifacts such as open web pages, documents, or calendar events. In that case, the critical difference to AffectAura would be this two-sided view and the possibility to share moods should one wish, or instead keep private for more self-reflection purposes. In addition, such an approach would allow for new research ideas such as comparing shared moods and not shared moods, which could be interesting for following up the negative moods and honesty results from this thesis.

\section{Task Awareness}
As mentioned in the previous section, P3 liked the idea of linking moods to tasks for self-reflection. Similarly, P5 liked the idea of sharing a task list to get a sense of team members' current workload in addition to the shared moods. Following the idea of integration with existing tools (see \autoref{section:better_integration_with_established_tools}), success and adoption would likely be highest if this feature worked seamlessly with existing task management software. At the same time, the core idea behind the AmbientTeams approach is that our focus is not on tasks, which raises the question of whether such a feature fits into our more social approach. We would argue that providing a simple, more well-being-focused measurement such as workload (e.g., the number of tasks currently assigned) could be a people-focused measurement that could nicely complement the moods already shared in AmbientTeams, and could potentially further raise awareness.

\section{Automated Mood Capturing}
We asked the participants during the final interview what they generally think about an automatic sharing of moods through the use of video input from the webcam. In our opinion, this would lead to real-time sharing of moods and possibly even increase the honesty and accuracy of the shared moods. Furthermore, such a feature could positively effect the long-term usage of AmbientTeams, as it requires little to none effort to share moods. However, four out of five participants (P1, P2, P3, and P4) mentioned concerns about their privacy and confidentiality if a tool constantly accessed the camera and shared moods automatically. While P5 felt the idea was very progressive, being able to turn off automatic capture would be mandatory. Regardless, if moods are automatically detected, there should always be a confirmation before sharing a mood with the entire team to ensure that nothing undesirable is shared (P4, P5).

Because of privacy concerns with webcam access, we think using other approaches based more on biometric sensing that can be used without exposing too much private information (such as skin conductivity or respiration \autocite{picard2001toward}) might make it easier to find participants for a study. Regardless of how emotions would be measured, it is probably reasonable to ask users what they want to share before sharing it.

\section{Threats to Validity}
\textit{External validity:} In our preliminary evaluation, one threat to external validity is the generalization of a single team to the entire population of remote teams. Therefore, to achieve better generalization beyond the setting in which we conducted the preliminary evaluation, the study should be repeated in other teams. Regarding the generalizability of the sample data, there are risks due to the small number of data points and the very short study period. The short evaluation period may have biased results because of the novelty effect of our tool. Finally, the ongoing COVID-19 pandemic limits the ability to generalize our results to a situation outside of a pandemic.

\medskip\noindent\textit{Internal validity:} There may have been a slight bias in the open coding of the interviews. This is because the author of this paper was heavily involved in the open coding of the interviews. However, we attempted to minimize this risk by bringing in another researcher who was also familiar with the project. Last, because one participant dropped out during the evaluation period, there is a chance that the feedback obtained during this preliminary evaluation was positively biased in favor of the tool. 
\chapter{Conclusion}
\label{chapter:conclusion}
After identifying the social challenges posed by remote work, AmbientTeams was developed to help knowledge workers address these issues, focusing on three main concepts: minimal design, focus on people and their well-being (via mood-based micro-blogging), and spontaneous interactions.

Complementing our initial research efforts, the interviews confirmed that there seems to be a need for an informal way of sharing information within knowledge work teams. Our approach aimed to help alleviate feelings of isolation in the workplace and communicate current social states, especially moods, with the team. The resulting research prototype, AmbientTeams, was used by a team consisting of five knowledge workers for five days. The usability questionnaire and interviews indicated that AmbientTeams was easy and intuitive to use, with the mood-sharing functionality being the most popular among participants. We then discussed the broader effects of AmbientTeams. We found that it 1) helped knowledge workers be aware of each other's moods and availability status, 2) got to know each other better, 3) enabled communication outside of AmbientTeams, and 4) spurred self-reflection on one's moods.

Further, we found that participants would reject automatic mood detection requiring constant camera access due to confidential data and privacy concerns. Nonetheless, other interesting directions for AmbientTeams were found and discussed. Possibilities to pursue in the future include focusing exclusively on the asynchronous exchange aspect and better integration with existing communication platforms.

\begin{appendices}
    \chapter{Consent Form}
\label{chapter:consent_form}
The consent form was handed to the participants before the kick-off meeting. During the kick-off meeting, they had the chance to ask any questions regarding the consent form.

\begin{figure}[h]
    \centering
    \includegraphics[width=\linewidth, page=1]{./documents/consent_form.pdf}
\end{figure}

\begin{figure}[h]
    \centering
    \includegraphics[width=\linewidth, page=2]{./documents/consent_form.pdf}
\end{figure}

\begin{figure}[h]
    \centering
    \includegraphics[width=\linewidth, page=3]{./documents/consent_form.pdf}
\end{figure}

\begin{figure}[h]
    \centering
    \includegraphics[width=\linewidth, page=4]{./documents/consent_form.pdf}
\end{figure}

\chapter{Study Instructions}
\label{chapter:study_instructions}

\begin{figure}[h]
    \centering
    \includegraphics[width=\linewidth, page=1]{./documents/study_instructions.pdf}
\end{figure}

\begin{figure}[h]
    \centering
    \includegraphics[width=\linewidth, page=2]{./documents/study_instructions.pdf}
\end{figure}

\begin{figure}[h]
    \centering
    \includegraphics[width=\linewidth, page=3]{./documents/study_instructions.pdf}
\end{figure}

\begin{figure}[h]
    \centering
    \includegraphics[width=\linewidth, page=4]{./documents/study_instructions.pdf}
\end{figure}

\chapter{Prestudy Questionnaire}
\label{chapter:prestudy_questionnaire}

\begin{figure}[h]
    \centering
    \includegraphics[width=\linewidth, page=1]{./documents/prestudy_questionnaire.pdf}
\end{figure}

\begin{figure}[h]
    \centering
    \includegraphics[width=\linewidth, page=2]{./documents/prestudy_questionnaire.pdf}
\end{figure}

\begin{figure}[h]
    \centering
    \includegraphics[width=\linewidth, page=3]{./documents/prestudy_questionnaire.pdf}
\end{figure}


\begin{figure}[h]
    \centering
    \includegraphics[width=\linewidth, page=4]{./documents/prestudy_questionnaire.pdf}
\end{figure}

\begin{figure}[h]
    \centering
    \includegraphics[width=\linewidth, page=5]{./documents/prestudy_questionnaire.pdf}
\end{figure}


\chapter{Poststudy Questionnaire}
\label{chapter:poststudy_questionnaire}

\begin{figure}[h]
    \centering
    \includegraphics[width=\linewidth, page=1]{./documents/poststudy_questionnaire.pdf}
\end{figure}

\begin{figure}[h]
    \centering
    \includegraphics[width=\linewidth, page=2]{./documents/poststudy_questionnaire.pdf}
\end{figure}

\begin{figure}[h]
    \centering
    \includegraphics[width=\linewidth, page=3]{./documents/poststudy_questionnaire.pdf}
\end{figure}

\begin{figure}[h]
    \centering
    \includegraphics[width=\linewidth, page=4]{./documents/poststudy_questionnaire.pdf}
\end{figure}

\begin{figure}[h]
    \centering
    \includegraphics[width=\linewidth, page=5]{./documents/poststudy_questionnaire.pdf}
\end{figure}

\chapter{Semi-Structured Interview Guide}
\label{chapter:interview_guide}
\begin{figure}[h]
    \centering
    \includegraphics[width=\linewidth, page=1]{./documents/Semi-Structured Interview Guide.pdf}
\end{figure}

\begin{figure}[h]
    \centering
    \includegraphics[width=\linewidth, page=2]{./documents/Semi-Structured Interview Guide.pdf}
\end{figure}

\begin{figure}[h]
    \centering
    \includegraphics[width=\linewidth, page=3]{./documents/Semi-Structured Interview Guide.pdf}
\end{figure}
\end{appendices}

\backmatter
\printbibliography

\end{document}
