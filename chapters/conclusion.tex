\chapter{Conclusion}
\label{chapter:conclusion}
After having identified the social challenges posed by remote work, we developed an approach to help knowledge workers address these issues. Our approach focused on three main concepts: a design that is unobtrusive, putting the focus on people and their well-being, and fostering informal and spontaneous interactions. Consequently, the key element of our approach was a mood-based micro-blogging solution that allows knowledge workers to share moods and other states with the team and visualizes them in an unobtrusive way. We then developed a research prototype, AmbientTeams, and evaluated it on a team consisting of five knowledge workers.

Complementing our initial research efforts, the interviews confirmed that there seems to be a need for an informal way of sharing moods within knowledge work teams. Our approach aimed to help alleviate feelings of isolation in the workplace and communicate current social states, especially moods, with the team. The resulting research prototype, AmbientTeams, was used by a team consisting of five knowledge workers for five days. The usability questionnaire and interviews indicated that AmbientTeams was easy and intuitive to use, with the mood-sharing functionality being the most popular among participants. We then discussed the broader effects of AmbientTeams. We found that it helped knowledge workers to 1) be aware of each other's moods and availability status, 2) get to know each other better, 3) enable communication outside of AmbientTeams, and 4) spur self-reflection on one's moods.

We also found that participants would reject automatic mood detection that requires constant access to the camera due to privacy concerns. Nonetheless, other interesting future directions for AmbientTeams were found and discussed. Possibilities to pursue in the future include conducting a more comprehensive study, focusing solely on the micro-blogging aspect, working on better integration with existing communication platforms, or shifting to a more self-reflection-based approach.