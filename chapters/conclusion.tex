\chapter{Conclusion}
\label{chapter:conclusion}
After identifying the social challenges posed by remote work, AmbientTeams was developed to help knowledge workers address these issues, focusing on three main concepts: minimal design, focus on people and their well-being (via mood-based micro-blogging), and spontaneous interactions.

Complementing our initial research efforts, the interviews confirmed that there seems to be a need for an informal way of sharing information within knowledge work teams. Our approach aimed to help alleviate feelings of isolation in the workplace and communicate current social states, especially moods, with the team. The resulting research prototype, AmbientTeams, was used by a team consisting of five knowledge workers for five days. The usability questionnaire and interviews indicated that AmbientTeams was easy and intuitive to use, with the mood-sharing functionality being the most popular among participants. We then discussed the broader effects of AmbientTeams. We found that it 1) helped knowledge workers be aware of each other's moods and availability status, 2) got to know each other better, 3) enabled communication outside of AmbientTeams, and 4) spurred self-reflection on one's moods.

Further, we found that participants would reject automatic mood detection requiring constant camera access due to confidential data and privacy concerns. Nonetheless, other interesting directions for AmbientTeams were found and discussed. Possibilities to pursue in the future include focusing exclusively on the asynchronous exchange aspect and better integration with existing communication platforms.