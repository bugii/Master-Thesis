\chapter{Discussion}
\label{chapter:discussion}
This chapter discusses the key findings from our preliminary evaluation and elaborates on possible future directions that our approach, AmbientTeams, could take.

\section{Ambient Window}
In contrast to our expectations, the ambient window was not used as the main window when interacting with AmbientTeams. The high usability scores that resulted from the SUS questionnaire show that AmbientTeams is perceived as intuitive and easy to use. 
%However, both the usage data and interviews point out that our novel visualization approach using a glanceable ambient view has not entirely been a success. As mentioned by participants, the ambient window could be further improved. 
However, some participants mentioned that they were annoyed of the ambient window, as it always was in the foreground. Further, we believe that another reason could be that the ambient window is a somewhat novel and unknown approach compared to more traditional applications. In fact, P5 stated exactly that: \textit{\enquote{I think it [reason for not using the ambient window] was some insecurity on my part}}. Therefore, a more extended period of usage and getting used to such a window might be required.

However, there were no complaints that the content displayed in the ambient window was distracting in any way, making us feel optimistic about the ambient window and its general appearance. The suggested improvements for better usability are all feasible and could be implemented with reasonable effort.

\section{(Mood) Awareness and Informal Interactions}
Our participants confirmed that they experience a lack of mood awareness and social interactions when working remotely, reinforcing our motivation for developing AmbientTeams. As the in-house breakroom is well attended and gives new employees the chance to meet each other, it suggests that such a breakroom concept is generally perceived as essential and partially explains why the breakroom integrated into AmbientTeams might not have been used. 

Since the mood sharing functionality was used more extensively, AmbientTeams managed to increase awareness, primarily mood awareness. Like \textcite{church2010study}, who stated that being aware of one's moods could act as a springboard for communication, we obtain similar results. While we have not been able to identify a whole lot of informal communication within AmbientTeams, we see an increase in typical communication applications such as MS Teams or Skype directly after AmbientTeams was used. While this does not necessairily guarantee that a conversation took place at all or that it is of informal nature, it seems that interactions immediately following AmbientTeams seem to take place on more informal platforms such as Skype, which still leaves us feeling positive. 

In contrast to the beliefs of \textcite{garcia1999emotional} that emotions would be helpful when it comes to collaborating more effectively, we cannot make any statements about our participants. However, our participants mentioned that they would be likely open to help should they notice negative moods of their co-workers. 

\section{Sharing Behavior}
Likewise, our mood-based micro-blogging approach seems to not have had an impact on any work-related awareness.
This is in contrast to the study performed on WeHomer \autocite{dullemond2013fixing}, which found that the content of the shared status messages was often personal or non-work related, we find exclusively non-work-related content. The reason could be that this is either 1) a company-specific outcome or 2) the lack of enough data points. Despite these concerns, however, we believe that one possible explanation is that our mood-centered visualization approach, coupled with the ephemeral nature of the displayed content, primarily supported non-work-related exchanges.

We believe that this is also the reason why moods were shared far more frequently than status messages. We hypothesize that the visual representation of AmbientTeams stirred the participants unconsciously towards sharing more moods and non-work-related information. Regardless of the reasons for such behavior, we see this as a win as it positions AmbientTeams in a somewhat niche sector of communication tools.

To our surprise, the reasons behind deciding which moods should be shared or which not, particularly negative moods, were often brought up in the final interviews. While \textcite{dullemond2013fixing} found similar results in terms that positive moods were shared more frequently, we were able to gather first insights into the reasons why people might be hesitant to share negative moods.

Sharing mood with core team

\section{More Extensive Study}
The first possible continuation of this work and the AmbientTeams approach is a more extensive study. Such a study would involve more teams, ideally teams that differ in various aspects such as industry, age of participants, and corporate culture. Based on the interviews, we have reason to believe that \textit{age} (P5) and \textit{company culture} (P2) may be predictors of willingness to share moods in the workplace. Also, the study design worked well in the small setting of this thesis. No significant problems were reported concerning the tool. For those reasons, we see no problem with conducting the same study (except for some minor adjustments to the interview questions and questionnaires) in a larger setting without changing or adjusting the functionality of AmbientTeams. However, inputs gathered from the small study also outline some potential future directions and further developments of the tool, AmbientTeams, which could be realized before continuing with a more extensive examination. We discuss those possible updates in the following sections.

% - P2: "sometimes you have this barrier to share the mood, so you have to establish this culture in the team, that it becomes a normal thing that everyone shares their angry or sad mood."

\section{Focus on Asynchronous Communication}
The lack of used synchronous communication features (video/audio calls) leads us to believe that the more realistic and promising approach would be to focus exclusively on the parts of asynchronous communication that are not yet integrated into existing communication tools (e.g., Slack, Microsoft Teams, Zoom.us). This change would mean that the main functionality of AmbientTeams would be limited to sharing moods and status messages. The functionality to nudge or directly notify other team members could also be retained as a potential communication trigger. Given our findings and the fact that most companies have established a communication culture using a software solution with advanced collaboration features, we believe that AmbientTeams should not compete with such tools but rather focus on what is different from them: mood sharing and informal status sharing. However, these are findings that we obtained from a small group of participants, all part of one company, and therefore cannot be generalized. Regardless, simplifying AmbientTeams would also have the benefit of making it easier for study participants to learn to use, as there are fewer features to learn and discover.

\section{Better Integration With Established Tools}
\label{section:better_integration_with_established_tools}
Participants P2, P3, and P5 indicated that they would prefer to have only one application for their team communications. P2 argued that a single tool would increase the likelihood and time that they would use AmbientTeams. We see two ways we could improve the use and user experience of AmbientTeams in the long term: 1) two-way synchronization of data between existing tools and AmbientTeams, or 2) complete relocation of AmbientTeams functionality to existing tools. The former means that AmbientTeams would remain a standalone desktop application and continue to benefit from the freedom it provides. It would use application programming interfaces (APIs) to push and pull updates to and from existing communication tools. Potential information that could be shared includes availability status and status messages. To maintain \enquote{one click on the avatar to start a call}, it could also leverage the more mature video conferencing capabilities of the existing solution for more seamless interaction between AmbientTeams, making it less like \enquote{just another tool} and more of a potential facilitator for using existing tools. The second measure would go in a completely different direction, essentially moving all of AmbientTeams' existing functionality into existing ecosystems like Microsoft Teams or Slack. While this would satisfy our participants' desire for a universal communication tool, we would also lose a lot of flexibility. The ambient window would have to go, and it's not yet clear how much of the functionality we could adopt. More research would need to be done on the capabilities of these established communication platforms before ultimately deciding on the better approach.

\section{Self-Reflection}
Feedback from P3, P4, and P5 told us that there is a genuine interest and potential benefit in reflecting on one's mood. Therefore, one possibility would be for AmbientTeams to move more towards self-reflection in the future. This could be achieved through various new or slightly modified existing features. For example, when selecting a mood, the user could be asked via emoticons if they would like to share the selected mood or update the local mood. A dashboard could then provide the user with various visualizations to reflect on moods, similar to AffectAura \autocite{mcduff2012affectaura}. Furthermore, P3 mentioned that linking tasks to moods would be of high interest. Again, this is similar to AffectAura's functionality of linking emotions to artifacts such as open web pages, documents, or calendar events. In that case, the critical difference to AffectAura would be this two-sided view and the possibility to share moods should one wish, or instead keep private for more self-reflection purposes. In addition, such an approach would allow for new research ideas such as comparing shared moods and not shared moods, which could be interesting for following up the negative moods and honesty results from this thesis.

\section{Task Awareness}
As mentioned in the previous section, P3 liked the idea of linking moods to tasks for self-reflection. Similarly, P5 liked the idea of sharing a task list to get a sense of team members' current workload in addition to the shared moods. Following the idea of integration with existing tools (see \autoref{section:better_integration_with_established_tools}), success and adoption would likely be highest if this feature worked seamlessly with existing task management software. At the same time, the core idea behind the AmbientTeams approach is that our focus is not on tasks, which raises the question of whether such a feature fits into our more social approach. We would argue that providing a simple, more well-being-focused measurement such as workload (e.g., the number of tasks currently assigned) could be a people-focused measurement that could nicely complement the moods already shared in AmbientTeams, and could potentially further raise awareness.

\section{Automated Mood Capturing}
We asked the participants during the final interview what they generally think about an automatic sharing of moods through the use of video input from the webcam. In our opinion, this would lead to real-time sharing of moods and possibly even increase the honesty and accuracy of the shared moods. Furthermore, such a feature could positively effect the long-term usage of AmbientTeams, as it requires little to none effort to share moods. However, four out of five participants (P1, P2, P3, and P4) mentioned concerns about their privacy and confidentiality if a tool constantly accessed the camera and shared moods automatically. While P5 felt the idea was very progressive, being able to turn off automatic capture would be mandatory. Regardless, if moods are automatically detected, there should always be a confirmation before sharing a mood with the entire team to ensure that nothing undesirable is shared (P4, P5).

Because of privacy concerns with webcam access, we think using other approaches based more on biometric sensing that can be used without exposing too much private information (such as skin conductivity or respiration \autocite{picard2001toward}) might make it easier to find participants for a study. Regardless of how emotions would be measured, it is probably reasonable to ask users what they want to share before sharing it.

\section{Threats to Validity}
\textit{External validity:} In our preliminary evaluation, a threat to external validity exists in the generalization of a single team to the entire population of remote teams. Therefore, to better generalize beyond the setting we performed the study in, the study should be repeated in other teams. Concerning the generalization of the sampled data, there are threats due to few data points and a very short study period. The short study period might have led to biased results due to the novelty effect of our tool. Last, the ongoing COVID-19 pandemic limits the possibility of generalizing our results to a non-pandemic situation.

\medskip\noindent\textit{Internal validity:} There might be a slight bias when open coding the interviews. This is because the author of this thesis was heavily involved in open-coding the interviews. However, we tried to minimize this risk by involving another researcher, who was also closely aware of the project. Last, because one participant dropped out during the evaluation period, there is a chance that the feedback obtained during this study was positively biased in favor of the tool. 