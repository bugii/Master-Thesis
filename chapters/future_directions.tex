\chapter{Future Directions}
Having analyzed the interviews, collected data, and the questionnaires, the following sections elaborate on possible future directions that our approach, AmbientTeams, could take.

\label{chapter:future_directions}

\section{More Extensive Study}
The first possible continuation of this work and the AmbientTeams approach is a more extensive study. Such a study would involve more teams, ideally teams that differ in various aspects such as industry, age of participants, and corporate culture. Based on the interviews, we have reason to believe that age (P5) and company culture (P2) may be predictors of willingness to share moods in the workplace. Also, the study design worked well in the small setting of this paper. No significant problems were reported concerning the tool. For those reasons, we see no problem with conducting the same study (except for some minor adjustments to the interview questions and questionnaires) in a larger setting without changing or adjusting the functionality of AmbientTeams. However, inputs gathered from the small study also outline some potential future directions and further developments of the tool, AmbientTeams, which could be realized before continuing with a more extensive examination. We discuss those possible updates in the following sections.

% - P2: "sometimes you have this barrier to share the mood, so you have to establish this culture in the team, that it becomes a normal thing that everyone shares their angry or sad mood."

\section{Focus on Asynchronous Communication}
The lack of used synchronous communication features (video/audio calls) leads us to believe that the more realistic and promising approach would be to focus exclusively on the parts of asynchronous communication that are not yet integrated into existing communication tools (e.g., Slack, Microsoft Teams, Zoom.us). This change would mean that the main functionality of AmbientTeams would be limited to sharing moods and status messages. The functionality to nudge or directly notify other team members could also be retained as a potential communication trigger. Given our findings and the fact that most companies have established a communication culture using a software solution with advanced collaboration features, we believe that AmbientTeams should not compete with such tools but rather focus on what is different from them: mood sharing and informal status sharing. Simplifying AmbientTeams would also have the advantage of making it easier for study participants to learn, as there are fewer features to learn and discover.

\section{Better Integration With Established Tools}
\label{section:better_integration_with_established_tools}
Participants P2, P3, and P5 indicated that they would prefer to have only one application for their team communications. P2 argued that a single tool would increase the likelihood and time they would use AmbientTeams. We see two ways we could improve the use and user experience of AmbientTeams in the long term: 1) two-way synchronization of data between existing tools and AmbientTeams, or 2) complete relocation of AmbientTeams functionality to existing tools. The former means that AmbientTeams would remain a standalone desktop application and continue to benefit from the freedom it provides. It would use application programming interfaces (APIs) to push and pull updates to and from existing communication tools. Potential information that could be shared includes availability status and status messages. To maintain \enquote{one click on the avatar to start a call}, it could also leverage the more mature video conferencing capabilities of the existing solution for more seamless interaction between AmbientTeams, making it less like \enquote{just another tool} and more of a potential facilitator for using existing tools. The second measure would go in a completely different direction, essentially moving all of AmbientTeams' existing functionality into existing ecosystems like Microsoft Teams or Slack. While this would satisfy our participants' desire for a universal communication tool, we would also lose a lot of flexibility. The Ambient window would have to go, and it's not yet clear how much of the functionality we could adopt. More research would need to be done on the capabilities of these established communication platforms before ultimately deciding on the better approach.

\section{Self-Reflection}
Feedback from P3, P5, and P4 told us that there is a genuine interest and potential benefit in reflecting on one's mood. Therefore, a future direction of AmbientTeams could be toward a greater emphasis on self-reflection. This could be achieved through various new or slightly modified existing features. For example, when selecting a mood, the user could be asked via emoticons if they would like to share the selected mood or update the local mood. A dashboard could then provide the user with various visualizations to reflect on moods, similar to AffectAura \autocite{mcduff2012affectaura}. Similar to AffectAura's functionality to link emotions to artifacts such as open web pages, documents, or calendar events, P3 mentioned that linking tasks to moods would be of high interest. In that case, the critical difference to AffectAura would be this two-sided view and the possibility to share moods should one wish, or instead keep private for more self-reflection purposes. In addition, such an approach would allow for new research ideas such as comparing shared moods and not shared moods, which could be interesting for following up the negative moods and honesty results from this thesis.

\section{Task Awareness}
As mentioned in the previous section, P3 liked the idea of linking moods to tasks for self-reflection. Similarly, P5 liked the idea of sharing a task list to get a sense of team members' current workload in addition to the shared moods. Following the idea of integration with existing tools (see \autoref{section:better_integration_with_established_tools}), success and adoption would likely be highest if this feature worked seamlessly with existing task management software. At the same time, the core idea behind the AmbientTeams approach is that our focus is not on tasks, which raises the question of whether such a feature fits into our more social approach. We would argue that providing a simple, more well-being-focused measurement such as workload (e.g., the number of tasks currently assigned) could be a people-focused measurement that could nicely complement the moods already shared in AmbientTeams, and could potentially further raise awareness.

\section{Automated Mood Capturing}
TBD
