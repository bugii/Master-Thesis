\chapter{Introduction}
\label{chapter:introduction}
Software development has become increasingly distributed in recent years \autocite{herbsleb2001global}. This development is caused by globalization trends \autocite{herbsleb2007global} and the increasing popularity of working from home \autocite{ecoWorkingFromHome2021}. Reasons for increasing work from home include a more flexible schedule, increased work productivity, and spending less time and money on commuting \autocite{flores2019understanding, mulki2009set}. Additionally, the increased flexibility and autonomy allows employees to more easily manage their family responsibilities and leads to higher job satisfaction and employee retention \autocite{mulki2009set, gajendran2007good, madsen2011benefits}.

However, global software development brings challenges; namely, coordination and collaboration in a remote setting become significantly more difficult \autocite{herbsleb2007global}. This is because communication, especially informal communication, is hampered \autocite{sengupta2006research, herbsleb2007global, hinds2005understanding}, and the group awareness required for successful coordination and collaboration is lacking \autocite{herbsleb2007global, gutwin2004group}.

Through these difficulties, spontaneous, more informal communication is more difficult in remote work, resulting in reduced spontaneous communication \autocite{kraut1988patterns, sengupta2006research, herbsleb2007global, hinds2005understanding}. This is not desirable because such spontaneous communication, such as \enquote{corridor talk} or \enquote{watercooler talk} can help news spread more quickly among teams \autocite{herbsleb2000distance}, or reduce coordination problems \autocite{herbsleb1999architectures} by gathering important background information that enables more effective teamwork \autocite{lanubile2007collaboration, herbsleb2001global}. Common problems related to communication are that remote team productivity is further reduced by technical problems encountered in computer-mediated communication \autocite{sengupta2006research}. Furthermore, the richness of communication is drastically reduced because written text is the most common communication medium used by remote software developers \autocite{gutwin2004group}. Text-based communication, however, does not have equal ability to convey rich information, resulting in, among other things, limited ability to convey emotional information \autocite{hook2008interactional}.

The lack of social interactions can lead to other interpersonal problems. Examples include difficulties in building trust, maintaining working relationships, or leading to not feeling connected to the team \autocite{comella2020revisiting, olson2006bridging}. Furthermore, a lack of social and emotional interactions can lead to workplace isolation \autocite{marshall2007workplace, gorlick2020productivity, mulki2009set}.

There is a large body of existing research focused on improving these coordination and communication challenges. This is usually done with task/coding-related awareness tools (\autocite{biehl2007fastdash, jakobsen2009wipdash}) or tools that primarily promote spontaneous communication through virtual offices (\autocite{sasaki1999video, lou2012presencescape}). Another concept aimed at promoting informal communication is microblogging in the workplace (e.g. \autocite{ebner2008microblogging, ehrlich2010microblogging, zhang2010case,dullemond2013fixing}). Since there seems to be evidence that software developers can find all the information they need to do their jobs even without task- or work-object-oriented approaches \autocite{gutwin2004group}, our approach builds primarily on existing microblogging approaches. Specifically, our approach builds upon the core idea of WeHomer \autocite{dullemond2013fixing}, which was the first to extend a micro-blogging approach with sentiment sharing. The motivation for sharing moods comes from \textcite{garcia1999emotional}, who argued that when you are aware of the emotional state of your employees and act accordingly, you get better results from collaborative work. We argue that such solutions focusing on Emotional Awareness are underrepresented by recent research. To make this point stronger, worrying figures relating to workers' well-being and mental health, stating that the pandemic has led to an increase in stress in 65.9\% of people 44.4\% report a decrease in mental health. As a result, AmbientTeams is a \enquote{people first} approach to sharing mood and status to provide important informal and social awareness information with the goal of reduced feelings of isolation when working remotely by encouraging more informal exchanges with team members. By reducing feelings of isolation in the workplace, we could improve individual well-being and the entire team's performance. This is because strong team cohesion has been shown to positively impact team effectiveness and productivity \autocite{carlson2017virtual}, and feeling disconnected from colleagues has been shown to decrease engagement in productive tasks \autocite{lostFocus2020}.

AmbientTeams differs from existing micro-blogging solutions in that it is more mood-oriented and minimalist in its functionality and design than WeHomer. This is due to the person-centered approach that AmbientTeams takes: The content of the textual information is not put in the foreground but is meant to complement the shared moods, visualized in avatars that are at the center of AmbientTeams. In terms of its appearance, AmbientTeams takes an unobtrusive and minimal approach, unlike existing virtual office approaches - hence the \enquote{ambient} in its name. Despite the goal of unobtrusiveness, AmbientTeams offers video conferencing built into the application. Our research aims to discover how AmbientTeams is being used, what knowledge workers are sharing, and the broader implications of such an approach. Thus, the research questions we sought to answer are:

% Somewhat concerning to us is the seemingly little effort to improve the situation for remote workers suffering from the perception of workplace isolation, a feeling that is referred to as 

% Consequently, there is a decrease in communication frequency when physical distance between co-workers is increased {\autocite{herbsleb2003empirical} because much of the information normally present in a co-located setting is no longer available (e.g., whether a co-worker is in an available and an interruptable state (\autocite{})). 

% The current COVID-19 pandemic makes research in the field of remote work even more important because the majority of managers expect to have more flexible work from home policies post-pandemic, and employees would like to continue working from home at least partially \autocite{msworkindexconnection}, making the topic very much relevant also after the pandemic.

% While working from home has numerous benefits, it also comes with a range of challenges. On the benefits side, employers can realize savings in real estate costs, and the employee can benefit from more flexible work hours and spending less time and money commuting \autocite{mulki2009set}. However, shared challenges from working from home are that communication is reduced \autocite{kraut1988patterns} and suffers in quality \autocite{mulki2009set}. More specifically, informal communication drastically reduces when working from home \autocite{hinds2005understanding}. This reduction of informal communication can lead to difficulties building trust, maintaining work relationships, or not feeling attached to the team \autocite{comella2020revisiting, olson2006bridging}. Another consequence of remote work is the feeling of workplace isolation \autocite{mulki2009set, marshall2007workplace}. The feeling of isolation leads to not knowing whom to turn to in case of a problem or not feeling part of the company and is said to be caused by missing support from co-workers and opportunities for social and emotional interactions in a team \autocite{marshall2007workplace}. The pandemic further reinforces this influence leading to almost 60\% feeling less connected to their co-workers compared to before the pandemic \autocite{msworkindexconnection}. Since strong team cohesion has been shown to have a positive impact on the team's effectiveness and productivity \autocite{carlson2017virtual}, and the feeling of being disconnected from colleagues have been shown to impede engaging in productive tasks \autocite{lostFocus2020}, the connectedness with the team is of particular interest to us.

% A lack of awareness causes the challenges of working remotely: less information about co-workers is exchanged, e.g., no or fewer cues are available to identify team members' interruptibility or emotional states. This missing information makes it a lot harder to find opportune moments to initiate a conversation because it is often unknown whether a person might be in a deep focus state or whether a person might be more than happy to chat. Informal communication is further challenged because serendipity is missing when working remotely because people no longer randomly bump into each other at the water cooler or the coffee machine. Therefore, improving awareness in the workplace is the foundation of our approach.

% While there are several prior approaches to improve awareness within teams by showing the current coding tasks and work items that others are working on \autocite{biehl2007fastdash, jakobsen2009wipdash}, they do not focus on the person behind that work item. They thus do not put teams into the center of attention. To make this point stronger, recent research shows concerning numbers in regards to workers' well-being and mental health, stating that the pandemic has led to an increase in stress for 65.9\% of people and 44.4\% reported a decrease in mental health \autocite{qualtricksmental}. Therefore, our concept, AmbientTeams, follows a different approach by \enquote{putting people first}. It includes an ambient always-on-top overview of the core team members and their moods, status messages, and other states. In addition to such a microblogging approach, where team members can share information about their moods (and potential context), we aim to study further possibilities to foster and motivate serendipitous, informal exchanges with the team.

% In the next chapter, existing approaches and their underlying concepts are discussed before introducing our approach and its differences. The resulting prototype is then introduced in \autoref{chapter:prototype} and analyzed in the scope of a preliminary evaluation in \autoref{chapter:preliminary_evaluation}.


\bigskip\noindent\textit{Information Sharing}

\smallskip\noindent\textit{RQ1}: Is there a need for sharing moods/states with team members, and what are the reasons?

\smallskip\noindent\textit{RQ2}: What are knowledge workers willing to share with their team?

\medskip\noindent\textit{Impacts}

\smallskip\noindent\textit{RQ3}: What are the effects of Ambient Teams?

\setlength{\leftskip}{0.5cm}
\smallskip\noindent\textit{RQ3.1}: Do mood and state sharing increase the awareness between team members, and how? What do they learn from each other?

\smallskip\noindent\textit{RQ3.2}: Does it make users feel better to share information with their team?

\smallskip\noindent\textit{RQ3.3}: Does it stress/relax users to see more about their team?

\smallskip\noindent\textit{RQ3.4}: Does AmbientTeams reduce the feeling of isolation in remote knowledge work teams?

\setlength{\leftskip}{0pt}

\medskip\noindent\textit{Tool usage and workflows}

\smallskip\noindent\textit{RQ4}: How do knowledge workers use and interact with AmbientTeams? How do they integrate it into existing workflows?

\bigskip\noindent To answer those questions, a small preliminary evaluation with five knowledge workers was conducted, who used AmbientTeams over a period of one week. Our participants confirmed the importance of being aware of their co-workers' moods, something that was also found by \textcite{garcia1999emotional, dullemond2013fixing} and which fundamentally motivated our approach. As a consequence, the mood-sharing functionality was the most popular feature among participants. Regarding the broader effects of AmbientTeams, we found that it 1) helped knowledge workers be aware of each other's moods and availability status, 2) got to know each other better, 3) enabled (non-work-related) communication outside of AmbientTeams, and 4) spurred self-reflection on one's moods. To summarize, the main contributions of this work include

\begin{enumerate}
    \item Development of a mood-sharing approach that allows moods to be shared with the team with minimal effort
    \item Insights into the mood and status sharing behaviors within knowledge work teams and the impact such sharing can have on personal relationships, workplace isolation, or collaboration
    \item Provision and initial testing of a study design that can be used for a more comprehensive study, along with suggestions for future features that resulted
\end{enumerate}
