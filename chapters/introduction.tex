\chapter{Introduction}
In Software Development working remotely has become very popular over the past years. Due to the  Covid-19 pandemic this trend has grown even stronger, forcing many companies and their employees to work from home.
Further, the majority of managers expect to have more flexible work from home policies post-pandemic and employees would like to continue working from home (at least part-time) \autocite{msworkindexconnection}, making the topic very much relevant also after the pandemic.

While working from home has numerous benefits, it also comes with a range of challenges. On the benefits side, employers can realize savings in real estate costs and the employee can benefit from more flexible work hours and spending less time and money commuting \autocite{mulki2009set}. However, common challenges from working from home are that communication is reduced \autocite{kraut1988patterns} and suffers in quality \autocite{mulki2009set}. More specifically, informal communication is drastically reduced when working from home \autocite{hinds2005understanding}. This can lead to difficulties building trust, maintaining work relationships, or not feeling attached to the team \autocite{comella2020revisiting, olson2006bridging}. Another consequence of remote work is the feeling of workplace isolation \autocite{mulki2009set, marshall2007workplace}. This feeling of isolation leads to not knowing who to turn to in case of a problem or not feeling part of the company and is said to be caused by missing support from co-workers and opportunities for social and emotional interactions in a team \autocite{marshall2007workplace}. The pandemic further reinforces this influence leading to almost 60\% feeling less connected to their co-workers compared to before the pandemic \autocite{msworkindexconnection}. Since strong team cohesion has been shown to have a positive impact on the team's effectiveness and productivity \autocite{carlson2017virtual}, and the feeling of being disconnected from collegues have been shown to be an impediment to engaging in productive tasks \autocite{lostFocus2020}, the connectness with the team is of particular interst to us.

The challanges of working remotely are caused by a lack of awareness: less information about co-workers is exchanged, e.g. no or less cues are available to identify the interruptibility or emotional states of team members. This makes it a lot harder to find opportune moments to initiate a conversation because it is often unknown whether a person might be in a deep focus state, or whether a person might be more than happy to chat. Informal communication is further challenged because serendipity is missing when working remotely because people are no longer randomly bumping into each other at the water cooler or the coffee machine. Therefore, improving awareness in the workplace is the foundation of our approach.

While there are several prior approaches with the goal of improving awareness within teams by showing the current coding tasks and work items that others are working on \autocite{biehl2007fastdash, jakobsen2009wipdash}, they don't focus on the person behind that work item and thus don't put teams into the center of attention. To make this point stronger, recent research shows concerning number in regards to workers' well-being and mental health stating that the pandemic has led to an increase in stress for 65.9\% of people and 44.4\% reported a decrease in mental health \autocite{qualtricksmental}. Therefore, our concept, Ambient Teams, follows a different approach by ``putting the people first''. It does so by showing ambient always-on-top overview of the core team members and their moods, status messages, and other states. Other approaches (e.g. \autocite{ehrlich2010microblogging, ebner2008microblogging, zhang2010case}) follow a similar philosophy by studying micro-blogging at the workplace, an easy way of broadcasting short texts. Their results look promising and indicate higher connectedness with team members. Extending the purely text-based microblogging systems, \textcite{dullemond2013fixing} developed a microblogging system that shares selected moods in addition. We use their idea as a foundation for our work, with the goal of studying the behavior of mood sharing when making it optional, something that was not done by \textcite{dullemond2013fixing}.

In addition to microblogging, we aim to study further possibilities to fostering and motivate serendipitous, informal exchanges with the team or individual team team members, while keeping interruptions at in-opportune moments low.

In the next chapter existing approaches are discussed before our approach and its differences are introduced. Our approach is then analyzed in the scope of a perliminary evaluation and its results are discussed.


