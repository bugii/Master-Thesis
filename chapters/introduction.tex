\chapter{Introduction}
\label{chapter:introduction}
Software development has become increasingly distributed in recent years \autocite{herbsleb2001global}. This development is caused by globalization trends \autocite{herbsleb2007global} and the increasing popularity of working from home \autocite{ecoWorkingFromHome2021}. Reasons for increasing work from home include a more flexible schedule, increased work productivity, and spending less time and money on commuting \autocite{flores2019understanding, mulki2009set}. Additionally, the increased flexibility and autonomy allows employees to more easily manage their family responsibilities and lead to higher job satisfaction and employee retention \autocite{mulki2009set, gajendran2007good, madsen2011benefits}.

However, global software development brings challenges, namely coordination and collaboration in a remote environment become much more difficult \autocite{herbsleb2007global}. One reason for this challenge is the lack of workplace or group awareness required for successful coordination and collaboration \autocite{herbsleb2007global, gutwin2004group}. Workplace awareness defines awareness that results from the real-time combination of elements that workers keep in mind when collaborating \autocite{gutwin1996workspace}. Such elements can be people, what they are working on, what are are planning on working on next, which objects they are using, and more \autocite{gutwin1996workspace}. Consequently, a large body of existing research focuses on improving these coordination challenges by increasing awareness amongst team members. Task/coding-related awareness tools (\autocite{biehl2007fastdash, jakobsen2009wipdash}) are very popular among these. However, there seems to be evidence that software developers can find all the information they need for their work even without task- or work-oriented approaches \autocite{gutwin2004group}, which is why the focus of our research is more on communication.

Due to the lack of awareness, spontaneous, more informal communication is more difficult in remote work, resulting in reduced spontaneous communication \autocite{kraut1988patterns, sengupta2006research, herbsleb2007global, hinds2005understanding}. This is not desirable, because spontaneous communication, such as \enquote{corridor or watercooler talk}, can help news spread faster among teams \autocite{herbsleb2000distance} or reduce some of the coordination problems \autocite{herbsleb1999architectures} mentioned above by gathering important background information that enables more effective teamwork \autocite{lanubile2007collaboration, herbsleb2001global}. Moreover, not only is spontaneous, informal communication impeded, but technical problems in computer-mediated communication generally occur, reducing the productivity of remote teams \autocite{sengupta2006research}. There is also research that finds a decrease in communication frequency when the physical distance between employees is increased \autocite{herbsleb2003empirical}, which is not surprising, given that informal communication accounts for about 85\% of all communication \autocite{kraut1990informal}. Furthermore, the richness of communication is drastically reduced, as written text is the most common communication medium used by remote software developers \autocite{gutwin2004group}. However, text-based communication does not have the same ability to convey rich information, resulting in, among other things, a limited ability to convey emotional information \autocite{hook2008interactional}.

The lack of such social interactions can lead to other interpersonal problems, such as difficulties in building trust, maintaining working relationships, or lead to not feeling connected to the team \autocite{comella2020revisiting, olson2006bridging}. Furthermore, a lack of social and emotional interactions can lead to workplace isolation \autocite{marshall2007workplace, gorlick2020productivity, mulki2009set}. This is critical since feeling disconnected from colleagues has been shown to decrease engagement in productive tasks \autocite{lostFocus2020}, and strong team cohesion has been shown to positively impact team effectiveness and productivity of a team \autocite{carlson2017virtual}. Existing research has thus also looked into ways of bringing more social interactions within a team into remote work. Virtual offices are an approach that is taken, both in research (e.g., \autocite{sasaki1999video, lou2012presencescape}) and commercially (e.g. Branch\footnote{\url{https://branch.gg}}, Reslash\footnote{\url{https://reslash.co}}, Wonder\footnote{\url{https://wonder.me}}, or Gather\footnote{\url{https://gather.town}}). Another, less intrusive concept aimed at promoting informal communication is micro-blogging in the workplace (e.g., \autocite{ebner2008microblogging, ehrlich2010microblogging, zhang2010case, dullemond2013fixing}). WeHomer, a micro-blogging tool introduced by \textcite{dullemond2013fixing}, was the first to extend a micro-blogging approach with mood sharing. Their motivation for sharing moods came from \textcite{garcia1999emotional}, who argued that being aware of the emotional state of your colleagues and acting accordingly leads to better collaborative work results. By developing and studying WeHomer, \textcite{dullemond2013fixing} found an increase in team-connectedness and easy access to otherwise hard to obtain information. Their findings and the fact that the COVID-19 pandemic has lead to worrying figures relating to workers' well-being and mental health, stating an increase in stress in 65.9\% of people, and 44.4\% reporting a decrease in mental health \autocite{mswellbeing}, has led us to develop AmbientTeams.

AmbientTeams is a desktop application that allows knowledge workers to add their most important team members and visualize them in a glanceable, transparent, and always-on-top window. It differs from existing micro-blogging solutions in that it is more person- and mood-centric, and the novelty of its user interface. This is achieved by making status messages optional and taking a more person-centric approach: the content of textual information is not put in the foreground but is meant to complement the shared moods. These moods are visualized in mood-adapted avatars, which are the center of AmbientTeams. While \textcite{dullemond2013fixing} provides the ability to respond to shared posts with comments, AmbientTeams provides simple response options such as direct messaging and video conferencing to allow for spontaneous interactions. The goal of AmbientTeams is to increase the sense of belonging in the team, and the goal of this thesis is to find out how AmbientTeams is being used, what its users are sharing, and what the broader implications of our approach are. Thus, the research questions we sought to answer are:

% Somewhat concerning to us is the seemingly little effort to improve the situation for remote workers suffering from the perception of workplace isolation, a feeling that is referred to as 

% Consequently, there is a decrease in communication frequency when physical distance between co-workers is increased {\autocite{herbsleb2003empirical} because much of the information normally present in a co-located setting is no longer available (e.g., whether a co-worker is in an available and an interruptable state (\autocite{})). 

% The current COVID-19 pandemic makes research in the field of remote work even more important because the majority of managers expect to have more flexible work from home policies post-pandemic, and employees would like to continue working from home at least partially \autocite{msworkindexconnection}, making the topic very much relevant also after the pandemic.

% While working from home has numerous benefits, it also comes with a range of challenges. On the benefits side, employers can realize savings in real estate costs, and the employee can benefit from more flexible work hours and spending less time and money commuting \autocite{mulki2009set}. However, shared challenges from working from home are that communication is reduced \autocite{kraut1988patterns} and suffers in quality \autocite{mulki2009set}. More specifically, informal communication drastically reduces when working from home \autocite{hinds2005understanding}. This reduction of informal communication can lead to difficulties building trust, maintaining work relationships, or not feeling attached to the team \autocite{comella2020revisiting, olson2006bridging}. Another consequence of remote work is the feeling of workplace isolation \autocite{mulki2009set, marshall2007workplace}. The feeling of isolation leads to not knowing whom to turn to in case of a problem or not feeling part of the company and is said to be caused by missing support from co-workers and opportunities for social and emotional interactions in a team \autocite{marshall2007workplace}. The pandemic further reinforces this influence leading to almost 60\% feeling less connected to their co-workers compared to before the pandemic \autocite{msworkindexconnection}. Since strong team cohesion has been shown to have a positive impact on the team's effectiveness and productivity \autocite{carlson2017virtual}, and the feeling of being disconnected from colleagues have been shown to impede engaging in productive tasks \autocite{lostFocus2020}, the connectedness with the team is of particular interest to us.

% A lack of awareness causes the challenges of working remotely: less information about co-workers is exchanged, e.g., no or fewer cues are available to identify team members' interruptibility or emotional states. This missing information makes it a lot harder to find opportune moments to initiate a conversation because it is often unknown whether a person might be in a deep focus state or whether a person might be more than happy to chat. Informal communication is further challenged because serendipity is missing when working remotely because people no longer randomly bump into each other at the water cooler or the coffee machine. Therefore, improving awareness in the workplace is the foundation of our approach.

% While there are several prior approaches to improve awareness within teams by showing the current coding tasks and work items that others are working on \autocite{biehl2007fastdash, jakobsen2009wipdash}, they do not focus on the person behind that work item. They thus do not put teams into the center of attention. To make this point stronger, recent research shows concerning numbers in regards to workers' well-being and mental health, stating that the pandemic has led to an increase in stress for 65.9\% of people and 44.4\% reported a decrease in mental health \autocite{qualtricksmental}. Therefore, our concept, AmbientTeams, follows a different approach by \enquote{putting people first}. It includes an ambient always-on-top overview of the core team members and their moods, status messages, and other states. In addition to such a microblogging approach, where team members can share information about their moods (and potential context), we aim to study further possibilities to foster and motivate serendipitous, informal exchanges with the team.

% In the next chapter, existing approaches and their underlying concepts are discussed before introducing our approach and its differences. The resulting prototype is then introduced in \autoref{chapter:prototype} and analyzed in the scope of a preliminary evaluation in \autoref{chapter:preliminary_evaluation}.


\bigskip\noindent\textit{Information Sharing}

\smallskip\noindent\textit{RQ1}: Is there a need for sharing moods/states with team members, and what are the reasons?

\smallskip\noindent\textit{RQ2}: What are knowledge workers willing to share with their team?

\medskip\noindent\textit{Impacts}

\smallskip\noindent\textit{RQ3}: What are the effects of Ambient Teams?

\setlength{\leftskip}{0.5cm}
\smallskip\noindent\textit{RQ3.1}: Do mood and state sharing increase the awareness between team members, and how? What do they learn from each other?

% \smallskip\noindent\textit{RQ3.2}: Does it make users feel better to share information with their team?
\smallskip\noindent\textit{RQ3.2}: Does sharing moods and status messages affect the sharing user?

% \smallskip\noindent\textit{RQ3.3}: Does it stress/relax users to see more about their team?

\smallskip\noindent\textit{RQ3.3}: Does AmbientTeams reduce the feeling of isolation in remote knowledge work teams?

\setlength{\leftskip}{0pt}

\medskip\noindent\textit{Tool Usage and Workflows}

\smallskip\noindent\textit{RQ4}: How do knowledge workers use and interact with AmbientTeams? How do they integrate it into existing workflows?

\bigskip\noindent To answer those questions, we conducted a preliminary evaluation with five knowledge workers who used AmbientTeams for one week. Our participants confirmed the importance of being aware of their co-workers' moods, something that was also found by \textcite{garcia1999emotional, dullemond2013fixing} and what fundamentally motivated our approach. Consequently, the mood-sharing functionality was the most popular feature among participants, primarily used without an attached status message. Regarding the broader effects of AmbientTeams, we found that it helped knowledge workers to 1) be aware of each other's moods and availability status, 2) get to know each other better, 3) enable communication outside of AmbientTeams, and 4) spur self-reflection on one's moods. To summarize, the main contributions of this work include

\begin{enumerate}
    \item Insights into mood and status sharing behaviors within knowledge work teams and the impact such sharing can have on personal relationships, workplace isolation, or collaboration
    \item Successful development of a glanceable, always-on-top status sharing window, and initial insights into the usability of such an approach
    \item Provision and initial application of a study design that can be used for a broader study, and resulting suggestions for future features
\end{enumerate}

We start with an overview of related work in \autoref{chapter:related_work} before elaborating our approach and its key concepts in more detail in \autoref{chapter:approach}. Subsequently, our research prototype and all its features are presented in \autoref{chapter:prototype}. The study design for the preliminary evaluation conducted can be found in \autoref{chapter:preliminary_evaluation} and the results in \autoref{chapter:results_and_discussion}. Last but not least, possible future directions of our approach are outlined in \autoref{chapter:future_directions}.