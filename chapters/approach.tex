\chapter{Approach}
\label{chapter:approach}
Existing approaches are were shown to be successful at increasing knowledge workers' team awareness on topics such as who their co-workers and teammates are, which tasks everyone is working on, and the progress they have made \autocite{biehl2007fastdash, jakobsen2009wipdash, cheng2003jazzing, deline2005easing}. However, they only cover a limited view of awareness by providing few social or emotional cues. Therefore, our work focuses on social, casual information exchanges to help remote teams facing challenges with workplace isolation, team awareness, informal communication, and well-being. We aim to tackle these issues by allowing knowledge workers to quickly learn about the availability, moods, and other states of their core team members in a lightweight, informal manner. The critical underlying concepts of our approach are elaborated in this chapter.

\section{Focus on People}
Remote workers fear being \enquote{out of sight, out of mind} \autocite{bailey1999advantages} and potentially suffer from the perception of workplace isolation \autocite{mulki2009set, marshall2007workplace}. Additionally, virtual workers might fear that their efforts are not recognized or valued as much as their co-located colleagues \autocite{cooper2002telecommuting}. Despite those facts, only a few approaches developed for use at the workplace seem to include social awareness, an essential type of awareness at the workplace \autocite{greenberg1996awareness}. Some, such as JazzBand and ContactMap \autocite{cheng2003jazzing, whittaker2004contactmap} take a similar visual approach to ours by visualizing individual team members and enabling communication between them. However, by being an Integrated Development Environment (IDE) plugin, we argue that JazzBand's resulting communication likely is work-related and only used during coding activities. Similarly, ContactMap facilitates email communication \autocite{whittaker2004contactmap}, a formal type of communication and thus being unlikely to include any form of social awareness. For those reasons, our approach does not focus on those types of team awareness and its implications for more effective and efficient collaboration, but rather the people behind those artifacts by representing different team members' social states to raise social awareness. One essential part of our people-centered approach is purely visual; avatars of the team members are prominently placed in an ambient manner, which visually emphasizes the people rather than work artifacts. Other social awareness information displayed by our approach is elaborated in the following section.

\section{Mood and Context Sharing}
To leverage the positive impact of micro-blogging on the feeling of connectedness among colleagues \autocite{dullemond2013fixing}, we use a micro-blogging approach with optional mood sharing. 
%Because topics that are blogged about are usually informal \autocite{ehrlich2010microblogging}, and mood sharing seems to act as a springboard for conversations according to \textcite{church2010study}, micro-blogging is one way to foster informal and spontaneous chats in our approach.
Existing micro-blogging tools explicitly designed for use at work lay the foundation of our approach and the information we want to visualize in our glanceable, always-on-top view (more details are given in \autoref{section:unobrusive_design}). However, micro-blogging is a purely text-based form of communication.
%As \textcite{garcia1999emotional} already mentioned in 1999, there is a need for emotional awareness inside groupware.
Similarly to \autocite{mora2011supporting}, we argue that the focus on mood awareness in a team is underrepresented in research, especially in a society where many are facing mental challenges caused by the global COVID-19 pandemic \autocite{mswellbeing}. 
%\textcite{saari2008mobile} developed a mobile application with mood-sharing features aimed at knowledge workers; however, we have not been able to find a study where potential use cases of their approach were observed. We will, in contrast, study both of our approach's usability and use cases in a preliminary evaluation. 
%Our approach features mood-based blogging, which can be used to provide additional awareness information. Extending the purely text-based micro-blogging systems, \textcite{dullemond2013fixing} developed a micro-blogging system that shares selected moods in addition.
In contrast to WeHomer, a micro-blogging tool developed by \textcite{dullemond2013fixing}, we make mood sharing optional and visually de-emphasize the shared status message compared to the moods. Last but not least, combining an ambient approach introduced in \autoref{section:unobrusive_design} with such micro-blogging functionality is a combination that has not yet, to our knowledge, been proposed in existing research. It should be noted that in \autoref{section:wellbeing} we briefly distinguished between emotions, moods, and feelings. For simplicity, and to be consistent with previous research from \textcite{dullemond2013fixing}, we use the terms \enquote{moods}, and \enquote{emotions} interchangeably, even though some of the available moods are arguably meant to be more short-term than others.

\section{Spontaneous Interactions}
Remote workers tend to desire the social interaction of informal chats and spontaneous discussions \autocite{cooper2002telecommuting}, which makes the fostering of those types of communication a goal of our approach. Further, spontaneous communication is crucial for achieving high productivity and social goals \autocite{kraut1990informal} such as developing work relationships \autocite{comella2020revisiting, olson2006bridging}. While the micro-blogging concept employed by our approach has the potential to increase spontaneous interactions, we also want to enable serendipitous moments (e.g., random \enquote{watercooler talks}) and quick one-on-one interactions. 

\medskip\noindent\textit{Serendipitous Interactions} \\
The goal is to mimic the watercooler in the office. Thus, simple signaling, similarly to just walking to the watercooler in an office and indicating to the other team members that you are now on a break, is required. This effortless signaling is motivated by \textcite{chang2007out}, who emphasize that initiating a conversation must be as simple as possible. 
%The same approach also applies to the possibility of speak to another random team member through a video or audio call. Should a user feel the need to talk to another team member, they can indicate that now would be an appropriate time for a short informal conversation. If other team members feel the same, two team members can randomly be paired up for a video call.

\medskip\noindent\textit{Direct Interactions} \\
Similar to \textcite{chang2007out}, we believe that initiating a conversation should be made as easy as possible. To this end, our approach allows users to respond directly to shared moods or status updates, something that was heavily used in WeHomer developed by \textcite{dullemond2013fixing}. We extend their functionality by exploring different ways to initiate direct contact with another person. 
%For scenarios where the user wants to react to a mood or status message shared by another team member or get another team member's attention, there is a way to send direct messages that are meant to be short-lived and informal. In cases where a direct message is not required, and the goal of a user is to get attention from another team member, the user can make use of the concept of \enquote{nudging}. This concept aims to help in cases of help-seeking, a known problem when working remotely \autocite{herbsleb2003empirical}.

\section{Unobtrusive Design}
\label{section:unobrusive_design}
By mimicking real offices, virtual office approaches, many of which have been released due to the COVID-19 pandemic, all have a significant downside: requiring a fair bit of user interaction due to the visually complex user interface. We argue that this adds much unnecessary overhead and reduces long-term usability. In contrast, there are exceptions, such as Tandem\footnote{\url{https://tandem.chat/}}, which takes a slightly different approach in that it is less playful and visually demanding than the other commercial tools. However, our approach goes a step further by introducing a glanceable, ambient view, which does not require significant, additional effort to be helpful. 
%In contrast to existing ambient approaches, which often include physical devices (e.g., \autocite{ downs2012ambient, alavi2012ambient, rocker2004using}), making them less suitable in a remote setting due to the size of the device or its invisibility to colleagues, we make use of a pure software solution. Regardless of its format, having a limited amount of information on an ambient display is critical for both not being interruptive and costly to use \autocite{dabbish2004controlling}.
Thus, we want to keep interactions lightweight and casual; hence the functionality is kept simple, maybe even limited, by design. The information shared and displayed is transient, meaning no chat history is available, making the tool essentially useless for formal communication and keeping the user interface as clean and straightforward as possible. In addition, our approach visually emphasizes the topicality of information displayed to avoid outdated data that clutters the user interface. Further, to minimize interruptions and distractions, targeted use of notifications and the ability not to be contacted and hide potential distractions are required.
