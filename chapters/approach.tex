\chapter{Approach}
\label{chapter:approach}

Our approach aims to help software developers in remote teams facing challenges with workplace isolation, team awareness, informal communication within their team, and well-being. We aim to tackle these issues by allowing knowledge workers to quickly learn about the availability, moods and emotions, and other states of their core team members. The critical underlying concepts of our approach are listed and explained in the following.

\medskip\noindent\textbf{Ambient always-on-top, people-centered team view} \\
At the core of our approach sits the decision to create an ambient tool that does not require significant, additional effort to be helpful. Having a limited amount of information on an ambient display is critical for both not being interruptive and costly to use \autocite{dabbish2004controlling}. With the help of the ambient display, the goal is to create a sense of presence within the team, even when working from different locations. By constantly showing the most central team members, the goal is to foster a sense of belonging.

Compared to existing awareness-increasing approaches focusing primarily on task-related awareness and their implications for more effective and efficient collaboration, our policy puts the humans behind the work items into the center of focus. By fostering more social communication, our approach does not conflict with or replace existing communication patterns established in a company. Instead, it aims to provide additional social information about the individual team members.

\medskip\noindent\textbf{Transience and topicality} \\
We want to keep interactions light-weight and casual, so the functionality is kept simple, maybe even limited, by design. The information shared and displayed will be transient. Following this approach brings the benefit of really fostering informal communication because there is no guarantee that messages will be read. In addition, our approach visually emphasizes the topicality of information displayed to avoid outdated data that clutters the user interface.

\medskip\noindent\textbf{Mood and context sharing} \\
As the primary text-based informal communication possibility, our approach focuses on status messages that can be broadcasted to the entire team. This concept is known as micro-blogging, with the difference that common micro-blogging platforms often are public (e.g., Twitter), and our approach is scoped to a team. \textcite{dullemond2013fixing} state that micro-blogging is another strategy that allowed software engineers to share activities and moods with other team members with the result of feeling more connected to each other. Our approach is very similar to theirs in terms of functionality of mood-based status sharing, yet we do not force the users to share moods accompanied by a status message; we are interested in use-cases for the combined and isolated usage of the two features. Further, topics that are usually blogged about are informal \autocite{ehrlich2010microblogging} which aligns with the goal of our approach. Furthermore, mood sharing seems to acts as a springboard for conversations according to \textcite{church2010study}, which suits our goal of fostering communication.

\medskip\noindent\textbf{Ever-running break room} \\
Additionally, allowing to see the team (and not just relying on text-based information) is possible by joining an ever-running break room. The goal is to mimic the water-cooler in the office. Thus, visiting a breakroom as simple as possible, similar to just walking to the coffee machine in an office and signaling to the other team members that you are now on a break, is required. This effortless joining of a breakroom is motivated by \textcite{chang2007out}, who emphasizes that initiating a conversation must be as simple as possible. This approach also applies to the next concept on the list, interactions that target individual team members.

\medskip\noindent\textbf{1:1 interactions} \\
For scenarios where the content you want to share is intended for a single person, or you want to get another team member's attention, there's an easy way to start a private conversation. This can be done through a direct message or by nudging a team member. This concept aims to help in cases of help-seeking, a known problem when working remotely \autocite{herbsleb2003empirical}. Recalling the transient nature of our approach, this communication mechanism is best suited for making a non-interruptive request that is not urgent. Should a user feel the need to talk to another team member, they can indicate that now would be an appropriate time for a short informal conversation. If other team members feel the same, two team members can randomly be paired up for a virtual video call.

\medskip\noindent\textbf{Minimizing interruptions} \\
When developing an ambient always-on-top visualization, minimizing interruptions and distractions is one of the most important design principles. Specifically, this means a very targeted use of notifications and the ability not to be contacted and hide potential distractions if desired.
