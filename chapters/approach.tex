\chapter{Approach}
\label{chapter:approach}
While the team awareness gained by existing tools allows knowledge workers to understand who they working with and what they are working on, and what the impact of a change can have on others, which is essential for successful collaboration \autocite{dourish1992awareness}, they only cover a very limited view of awareness by providing very limited social or emotional information. Because of that, we put more emphasis on social, casual information exchanges to help remote teams facing challenges with workplace isolation, team awareness, informal communication within their team, and well-being. We aim to tackle these issues by allowing knowledge workers to quickly learn about the availability, moods, and other states of their core team members in a lightweight, informal manner. The critical underlying concepts of our approach are elaborated in the following.

\section{Minimal design}
By mimicking real offices, virtual office approaches, which have been coming out a lot due to the COVID-19 pandemic, all have a significant downside: requiring a fair bit of user interaction due to the visually complex interface. We argue that this adds a lot of unnecessary overhead and reduces long-term usability. In contrast, there are exceptions, such as Tandem\footnote{\url{https://tandem.chat/}}, which takes a slightly different approach in that it is less playful and visually demanding than the other commercial tools. However, our approach goes a step further by introducing a glanceable, ambient view, which does not require significant, additional effort to be helpful. Having a limited amount of information on an ambient display is critical for both not being interruptive and costly to use \autocite{dabbish2004controlling}. Thus, we want to keep interactions lightweight and casual, so the functionality is kept simple, maybe even limited, by design. The information shared and displayed will be transient, meaning that there will be no chat history available, making the tool essentially unuseful for formal communication and keeping the user interface as clean and straightforward as possible. In addition, our approach visually emphasizes the topicality of information displayed to avoid outdated data that clutters the user interface. Further, to minimize interruptions and distractions, targeted use of notifications and the ability not to be contacted and hide potential distractions is required. What's more, many existing ambient solutions include physical devices (e.g., \autocite{ downs2012ambient, alavi2012ambient, rocker2004using}), which might not be suitable for a remote team setting due to the size of the device or the device with the awareness information would not be visible to off-site team members.

\section{People-centeredness}
Remote workers fear being "out of sight and out of mind" \autocite{bailey1999advantages} and potentially suffer from the perception of workplace isolation \autocite{mulki2009set, marshall2007workplace}. Additionally, virtual workers tend to believe that their efforts are not recognized or valued as much as their co-located colleagues \autocite{cooper2002telecommuting}. Despite those facts, existing ambient approaches developed for use at the workplace don't seem to focus on social awareness, an essential type of awareness at the workplace \autocite{greenberg1996awareness}. Some, such as JazzBand and ContactMap \autocite{cheng2003jazzing, whittaker2004contactmap} follow similar principles by visualizing individual team members. However, by being an Integrated Development Environment (IDE) plugin, we argue that JazzBand's resulting communication likely is work-related and only used when coding and limited to software developers. Similarly, ContactMap facilitates email communication, a formal type of communication and thus being unlikely to include any form of social awareness. For those reasons, our approach does not focus on task-related awareness and its implications for more effective and efficient collaboration, but rather the humans behind those tasks by representing different team members' social states to raise awareness for feelings of isolation.

One essential part of our people-centered approach is purely visual; avatars of the team members are prominently placed in an ambient manner, as explained above.

\medskip\noindent\textbf{Mood and context sharing}\\
To leverage the positive impact of microblogging on the feeling of connectedness among colleagues \autocite{dullemond2013fixing}, the users can share their feelings with their colleagues through microblogging with optional mood sharing. Micro-blogging tools designed specifically for use at work lay the foundation of our approach and the information we want to visualize in our glanceable, always-on-top view. However, microblogging is a purely text-based form of communication. As \textcite{garcia1999emotional} already mentioned in 1999, there is a need for emotional awareness inside groupware. Therefore, and similar to \autocite{mora2011supporting}, we argue the awareness of moods in a work environment is underrepresented in research, especially in a society where many are facing mental challenges caused by the global COVID-19 pandemic. Extending the purely text-based microblogging systems, \textcite{dullemond2013fixing} developed a microblogging system that shares selected moods in addition. We use their idea as a foundation for our work, to study the behavior of mood sharing when making it optional, something not done by \textcite{dullemond2013fixing}. Last but not least, combining an ambient approach introduced above with such micro-blogging functionality is a combination that has not yet, to our knowledge, been proposed in existing research. In contrast to \textcite{saari2008mobile}, who developed another mobile application with mood sharing features aimed at knowledge workers, we will study both of our approach's usability and use cases in a preliminary evaluation.

\section{Casual Discussions}
Remote workers miss the social interaction of informal chats and spontaneous discussions \autocite{cooper2002telecommuting}, which makes the fostering of those types of communication a goal of our approach. By following our social, people-centered approach, the goal is to create possibilities for teams to share and discuss information in a casual, informal manner. While topics that are usually blogged about are informal \autocite{ehrlich2010microblogging}, and mood sharing seems to acts as a springboard for conversations according to \textcite{church2010study}, the micro-blogging functionality introduces above may already have a positive effect on more informal and spontaneous chats. To further foster and allow such conversations to occur, our approach also offers additional functionality, namely an ever-running break room and 1:1.

\medskip\noindent\textbf{Ever-running break room} \\
Allowing to see the team, and not just relying on text-based information, is possible by joining an ever-running break room. The goal is to mimic the water-cooler in the office. Thus, visiting a breakroom as simple as possible, similar to just walking to the coffee machine in an office and signaling to the other team members that you are now on a break, is required. This effortless joining of a breakroom is motivated by \textcite{chang2007out}, who emphasizes that initiating a conversation must be as simple as possible. This approach also applies to the next concept on the list, interactions that target individual team members.

\medskip\noindent\textbf{1:1 interactions} \\
For scenarios where the content you want to share is intended for a single person, or you want to get another team member's attention, there's an easy way to start a private conversation. This can be done through a direct message or by nudging a team member. This concept aims to help in cases of help-seeking, a known problem when working remotely \autocite{herbsleb2003empirical}. Recalling the transient nature of our approach, this communication mechanism is best suited for making a non-interruptive request that is not urgent. Should a user feel the need to talk to another team member, they can indicate that now would be an appropriate time for a short informal conversation. If other team members feel the same, two team members can randomly be paired up for a virtual video call.

\section{Research Questions}
Following the above-outlined concepts, we developed AmbientTeams, a research prototype introduced in the next chapter. To evaluate, three main areas of interest and the research questions we would like to answer in the scope of this master thesis are as follows.

\begin{itemize}
      \item Information Sharing\\
            RQ1: Is there a need for sharing moods/states with team members, and what are the reasons? (e.g., share your status with them to indicate states or know more about your team)\\
            RQ2: What are knowledge workers willing to share with their team? (is that impacted by what others on the team are sharing?)
      \item Impacts\\
            RQ3: What are the effects of Ambient Teams?\\
            RQ3.1: Do mood and state sharing increase the awareness between team members, and how? What do they learn from each other?\\
            RQ3.2: Does it make users feel better to share information with their team?\\
            RQ3.3: Does it stress/relax users to see more about their team?\\
            RQ3.4: Does AmbientTeams reduce the feeling of isolation in remote knowledge work teams?
      \item Tool usage and workflows\\
            RQ4: How do knowledge workers use and interact with AmbientTeams? How do they integrate it into existing workflows?
\end{itemize}
