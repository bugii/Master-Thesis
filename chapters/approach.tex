\chapter{Approach}
\label{chapter:approach}
While the team awareness gained by existing tools allows knowledge workers to understand who they are working with and what they are working on, and what the impact of a change can have on others, which is essential for successful collaboration \autocite{dourish1992awareness}, they only cover a very limited view of awareness by providing very limited social or emotional information. Because of that, we put more emphasis on social, casual information exchanges to help remote teams facing challenges with workplace isolation, team awareness, informal communication within their team, and well-being. We aim to tackle these issues by allowing knowledge workers to quickly learn about the availability, moods, and other states of their core team members in a lightweight, informal manner. The critical underlying concepts of our approach are elaborated in the following.

\section{Unobtrusive Design and Glancable Window}
By mimicking real offices, virtual office approaches, which have been coming out a lot due to the COVID-19 pandemic, all have a significant downside: requiring a fair bit of user interaction due to the visually complex interface. We argue that this adds a lot of unnecessary overhead and reduces long-term usability. In contrast, there are exceptions, such as Tandem\footnote{\url{https://tandem.chat/}}, which takes a slightly different approach in that it is less playful and visually demanding than the other commercial tools. However, our approach goes a step further by introducing a glanceable, ambient view, which does not require significant, additional effort to be helpful. Having a limited amount of information on an ambient display is critical for both not being interruptive and costly to use \autocite{dabbish2004controlling}. Thus, we want to keep interactions lightweight and casual, so the functionality is kept simple, maybe even limited, by design. The information shared and displayed will be transient, meaning that there will be no chat history available, making the tool essentially unuseful for formal communication and keeping the user interface as clean and straightforward as possible. In addition, our approach visually emphasizes the topicality of information displayed to avoid outdated data that clutters the user interface. Further, to minimize interruptions and distractions, targeted use of notifications and the ability not to be contacted and to hide potential distractions is required. What's more, many existing ambient solutions include physical devices (e.g., \autocite{ downs2012ambient, alavi2012ambient, rocker2004using}), which might not be suitable for a remote team setting due to the size of the device or the device with the awareness information not being visible to off-site team members.

\section{Focus on People}
Remote workers fear being \enquote{out of sight and out of mind} \autocite{bailey1999advantages} and potentially suffer from the perception of workplace isolation \autocite{mulki2009set, marshall2007workplace}. Additionally, virtual workers might fear that their efforts are not recognized or valued as much as their co-located colleagues \autocite{cooper2002telecommuting}. Despite those facts, existing ambient approaches developed for use at the workplace don't seem to focus on social awareness, an essential type of awareness at the workplace \autocite{greenberg1996awareness}. Some, such as JazzBand and ContactMap \autocite{cheng2003jazzing, whittaker2004contactmap} follow similar principles by visualizing individual team members. However, by being an Integrated Development Environment (IDE) plugin, we argue that JazzBand's resulting communication likely is work-related and only used when coding and limited to software developers. Similarly, ContactMap facilitates email communication, a formal type of communication and thus being unlikely to include any form of social awareness. For those reasons, our approach does not focus on work artifact-based awareness and its implications for more effective and efficient collaboration, but rather the people behind those artifacts by representing different team members' social states to raise social awareness. One essential part of our people-centered approach is purely visual; avatars of the team members are prominently placed in an ambient manner, which visually focuses on the people rather than work artifacts. Other social awareness information displayed by our approach is elaborated in the following section.

\section{Mood and Context Sharing}
To leverage the positive impact of microblogging on the feeling of connectedness among colleagues \autocite{dullemond2013fixing}, the users can share their feelings with their colleagues through microblogging with optional mood sharing. Because of the fact that topics that are blogged about are usually informal \autocite{ehrlich2010microblogging}, and mood sharing seems to act as a springboard for conversations according to \textcite{church2010study}, microblogging is one way to foster informal and spontaneous chats in our approach. Existing microblogging tools designed specifically for use at work lay the foundation of our approach and the information we want to visualize in our glanceable, always-on-top view. However, microblogging is a purely text-based form of communication. As \textcite{garcia1999emotional} already mentioned in 1999, there is a need for emotional awareness inside groupware. Therefore, and similar to \autocite{mora2011supporting}, we argue that the focus on mood awareness in a team is underrepresented in research, especially in a society where many are facing mental challenges caused by the global COVID-19 pandemic. In contrast to \textcite{saari2008mobile}, who developed mobile application with mood-sharing features aimed at knowledge workers, no study was conducted and potential use cases observed. We will, in contrast, study both of our approach's usability and use cases in a preliminary evaluation. Additionally, our approach also features text-based blogging, which can be used to provide additional awareness information. Extending the purely text-based microblogging systems, \textcite{dullemond2013fixing} developed a microblogging system that shares selected moods in addition. We use their idea as a foundation for our work, to study the behavior of mood sharing when making it optional, something not done by \textcite{dullemond2013fixing}. Last but not least, combining an ambient approach introduced above with such micro-blogging functionality is a combination that has not yet, to our knowledge, been proposed in existing research.

\section{Spontaneous Interactions}
Remote workers miss the social interaction of informal chats and spontaneous discussions \autocite{cooper2002telecommuting}, which makes the fostering of those types of communication a goal of our approach. While the microblogging concept employed by our approach has to potential to increase spontaneous interactions, our approach also offers additional functionality, namely an ever-running breakroom and quick one-on-one interactions, to further foster and allow such conversations to occur.

\medskip\noindent\textit{Ever-Running Breakroom and Random Video Calls} \\
Allowing to see the team, and not just relying on text-based information, is possible by joining an ever-running breakroom. The goal is to mimic the water-cooler in the office. Thus, visiting a breakroom as simple as possible, similar to just walking to the coffee machine in an office and signaling to the other team members that you are now on a break, is required. This effortless joining of a breakroom is motivated by \textcite{chang2007out}, who emphasize that initiating a conversation must be as simple as possible. This approach also applies to the possibility of speak to another random team member through a video or audio call. Should a user feel the need to talk to another team member, they can indicate that now would be an appropriate time for a short informal conversation. If other team members feel the same, two team members can randomly be paired up for a virtual video call.

\medskip\noindent\textit{Direct Interactions} \\
For scenarios where you want to react to a mood or status message shared by another team member, or you want to get another team member's attention, there's a way to send direct messages that are meant to be short-lived and informal. In cases where a direct message is not required, and the goal of a user is to get attention from another team member, the user can make use of the concept of \enquote{nudging}. This concept aims to help in cases of help-seeking, a known problem when working remotely \autocite{herbsleb2003empirical}.
