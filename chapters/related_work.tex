\chapter{Related Work}
Remote work offers numerous benefits for both the employee and employer compared to traditional co-located work. Benefits on the employee side include a more flexible schedule, higher job productivity, and less time and money spent commuting \autocite{flores2019understanding, mulki2009set}. The increased flexibility and autonomy allows employees to more easily deal with their family responsibility and leads to higher levels of job satisfaction and higher employee retention \autocite{mulki2009set, gajendran2007good, madsen2011benefits}, both highly beneficial for the employer. The employer can further profit from savings in real estate costs and increased productivity \autocite{mulki2009set}. In addition to those general benefits, there is another popular reason for building distributed teams: the possibility to build teams with talents from all over the world \autocite{carmel1999global}.

However, remote work creates new challenges for the company and its employees. Therefore, it is not surprising that much research has been done in this area, most of which coming from Computer-Supported Collaborative Work (CSCW). The general goal of existing solutions is to support distributed teams in accomplishing work as effectively and efficiently as possible. While a lot of research goes into collaboration and coordination challenges in remote work, the goal of AmbientTeams is fostering social, informal interactions. As a result, we identified four main social challenges that are the result of working remotely. Together with existing solutions aiming at solving those problems, those four challenges are are discussed in this chapter.

% We categorize the main challenges of remote work into the following three distinct yet very related and interdependent categories: awareness, communication, and well-being. Keeping the challenges of working remotely in mind, we first take a closer look at those three concepts and then take a look at the existing approaches to solve those problems.

% \textbf{Interruptions and Work Fragmentation}\\
% Interruptions at the workplace have been shown to decrease productivity because when being interrupted, users require more time to complete tasks, make more errors, and experience more annoyance \autocite{bailey2006need}. Due to the lack of attentional states available when working remotely, interruptions cannot be timed as carefully \autocite{mark2005no}.

\section{Workplace Isolation}
\textcite{marshall2007workplace} define workplace isolation as the "[...] desire to be part of the network of colleagues who provide help and support in specific work-related needs. It represents employees' perceptions of availability of co-workers, peers, and supervisors for work-based social support." They further suggest a categorization into social isolation and organizational isolation. Oranzisational isolation stems from the perception that remote workers might feel "out of sight, out of mind" \autocite{bailey1999advantages}, while social isolation relates to the fact that remote workers miss informal chats, spontaneous discussions, and meetings around the water cooler \autocite{cooper2002telecommuting}. For those reasons, a closer look at communication and, more specifically, informal communication will be given in the following.

\section{Communication}
%Modularisation of source code is a prevalent design principle in the context of software engineering. However, software development is considered to be a highly complex task, meaning that even in very well-designed systems with high modularisation, some dependencies remain \autocite{cataldo2007coordination}. Many studies have shown that successful coordination during the process of software development leads to higher performance \autocite{kraut1995coordination}. Distributed teams have a disadvantage coordinating compared to traditional collocated teams \autocite{herbsleb2003empirical}.
Research in the field of software development states that co-workers are the most used source of information used by developers \autocite{ko2007information}, emphasizing the importance of team communication inside software development teams. When shifting from traditional, co-located work to remote work, studies find different results regarding the communication frequency. While \textcite{kraut1988patterns, allen1984managing} find a decrease in communication, \textcite{mulki2009set} find increased communication in a remote setting. A possible reason for more communication includes the need for remote workers to over-communicate their availability status to their co-workers \autocite{koehne2012remote}. Reasons for communication reduction could be the active effort to bring back ad-hoc meetings \autocite{miller2021your}, or the lack of the required awareness to initiate a conversation. Regardless of communication frequency, working remotely and thus using software to communicate leads to having more misunderstandings due to missing cues and thus reduces communication effectiveness \autocite{mulki2009set}. Other researchers argue that the reason for those misunderstandings is the fact text-based communication (which is often used in software development) has very limited capacity, and thus a lot of socio-emotional information (non-verbal cues) is missing \autocite{hassib2017heartchat}. This most likely is a reason why face-to-face communication is still very important for many developers \autocite{storey2016social} and a lack thereof, which is caused by working remotely, can lead to workplace isolation making it harder to develop personal relationships and build trust \autocite{mulki2009set}. \textcite{gajendran2007good} state that working from home with high-intensity (more than 2.5 days a week) harmed relationships with co-workers, something that is enforced because of the Covid-19 pandemic. Since informal communication helps developing work relationships \autocite{comella2020revisiting, olson2006bridging}, it is of special importance in distributed teams.


% This is likely more pronounced because of the global pandemic, which forced teams to switch from collocated to remote work, making communication as accessible as possible should be desirable regardless.


\subsection{Informal Communication}
\textcite{kraut1990informal} define informal communication as "[...] communication that is spontaneous, interactive and rich". Differences to formal communication include lack of planning and the fact that the content of the communication is unknown in advance. \textcite{kraut1990informal} further state that over 85\% of all conversations are informal and informal communication happens more often if there is a short physical distance between parties. Similarly, \textcite{hinds2005understanding} find that members of distributed teams engage less in informal conversations. This reduction of informal communication is unfortunate since informal communication is crucial for achieving high productivity and social goals \autocite{kraut1990informal} such developing work relationships \autocite{comella2020revisiting, olson2006bridging}. More concretely, in the field of software development, informal communication plays a critical role due to the fast speed at which informal communication distributes knowledge across a team or company \autocite{french1998study, mockus2001challenges}. Also, informal communication can increase awareness, enabling developers to work efficiently \autocite{herbsleb2001global}. In the ever-changing field of agile software development, this is particularly useful because requirements can change, and formal communication channels cannot spread the news as fast. Besides, informal communication is essential for conflict identification and handling \autocite{hinds2005understanding}. The fact that teams with a high degree of social interactions often have better team cohesion \autocite{staehle2014management}, and informal communication is normally much more frequent than formal forms of communication \autocite{kraut1990informal}, further pronounces the importance of informal communication.

Because of those benefits, it is no surprise that numerous approaches are fostering informal communication inside distributed teams. One of the earliest proposed solutions for promoting informal communication in distributed teams was VideoWindow \autocite{fish1990videowindow}. Despite being an early solution, the authors already identified two essential requirements such a system must offer: low personal cost and the need for a visual channel. Suppose the prices for initiating conversations are too high. In that case, the system will not be helpful because the tool will not be used. The visual channel also plays a vital role by recognizing the presence of other people, indicating whether a conversation can be initiated. \textcite{sasaki1999video} developed a hallway system that was able to raise awareness and helped to indicate that one might have a question but failed to promote casual interactions. In comparison, \textcite{lou2012presencescape} manages to provide awareness information that is relevant to engage in everyday conversations and a low-effort mechanism to initiate such informal discussions. It does so by providing social cues which help understand the availability of others and thus creating a context for subsequent communication.

As a consequence of the global pandemic, many commercial tools have been published recently. Branch\footnote{\url{https://branch.gg}}, Reslash\footnote{\url{https://reslash.co}}, Wonder\footnote{\url{https://wonder.me}}, or Gather\footnote{\url{https://gather.town}} also follow the goal of increasing spontaneous, informal communication by creating virtual offices where users can move around with avatars and interact with others. Tandem\footnote{\url{https://tandem.chat/}} is another tool with a focus on collaboration and takes a less playful approach by being more similar to traditional communication apps user interfaces.

Another form of communication that has been studied extensively is the concept of microblogging. Studies have shown that microblogging is a form of informal communication \autocite{ehrlich2010microblogging} that is "[...] like a virtual coffee machine as a meeting place" \autocite{ebner2008microblogging}. Further, many existing microblogging approaches have found that microblogging results in people feeling more connected \autocite{ehrlich2010microblogging, zhang2010case}. Likewise, their study participants found microblogging very helpful because it allowed them to stay aware of what their team members are doing \autocite{zhang2010case}. In addition to purely share text-based contents, which is the standard in microblogging, \textcite{dullemond2013fixing} developed a microblogging system that allows the users to attach a mood to each message which helped the teams feeling more connected. What they did not measure, however, is the isolated effect of mood sharing.

Due to the value of providing additional awareness and sharing moods in the workplace, the following two sections focus on those two concepts.

\section{Awareness}
A reason for coordination and communication challenges in a remote work environment is the lack of awareness, so it is of great interest to increase awareness in distributed teams. \textcite{gutwin1996workspace} defines group awareness as a combination of:

\begin{itemize}[itemsep=0ex, parsep=0ex, leftmargin=*]
      \item \textbf{Informal Awareness} \\
            The general sense of the presence, availability, and activities of others. It is the "glue that facilitates casual interactions" \autocite{gutwin1996workspace}.
      \item \textbf{Group-Structural Awareness} \\
            "Group-structural awareness involves the knowledge about people's roles and responsibilities, their positions on an issue, their status, and group processes" \autocite{gutwin1996workspace}.
      \item \textbf{Social Awareness} \\
            "Social awareness is the information that a person maintains about others in a social or conversational context "\autocite{gutwin1996workspace}. It includes things as the attention state of the other person, their emotions, or the level of interest \autocite{gutwin1996workspace}, or whether a person can be disturbed \autocite{gutwin1995support}.
      \item \textbf{Workplace Awareness} \\
            Defines the awareness that results from the 'real-time' combination of elements workers keep track of when working together. Such elements could be people, actions, objects, and many more \autocite{gutwin1995support}.
\end{itemize}

It is important to note that those four awareness types are not excluding but rather overlapping with each other. Put differently, informal, social, and group-structural awareness are all part of workplace awareness. In the case of software developers, for instance, a study shows that developers checked the availability status of their co-workers almost as many times as their compiler output \autocite{ko2007information}. This indicates the importance of informal awareness. Providing group-structural is essential because of difficulties when trying to find experts in a distributed team \autocite{herbsleb2003empirical}. Social awareness is highly relevant due to the high communication needs of software developers \autocite{perry1994people}. Additionally, with less face-to-face communication and more computer-mediated communication, it is consequently more difficult to transfer emotional information \autocite{rivera1996effects}.

Despite the seemingly precise categorization of awareness above, the literature does not agree on one common categorization. Alternative terms such as general awareness, peripheral awareness, co-existent and cooperation awareness, and objective self-awareness are used to describe and categorize awareness. Due to the popularity and granularity of the above definition, it is the definition of choice for this work.

To address the problem of missing awareness when working remotely, a wealth of research developed approaches to increase awareness in distributed teams. Popular tools made explicitly for software development teams focus on providing awareness by on work items, developers' activities (e.g., which files they have opened or recently changed) and thus put the code base and tasks in the foreground of coordination \autocite{biehl2007fastdash, jakobsen2009wipdash, eick1992seesoft, deline2005easing}. \textcite{cheng2003jazzing} introduces JazzBand, an IDE plugin visualizing the team members to increase peripheral awareness enhanced with status messages and chat functionality facilitating coordination.

While the majority of the above-mentioned awareness tools require user interactions to be helpful, there have also been attempts for creating ambient approaches to raise awareness in the work environment \autocite{morrison2020facilitating, otjacques2006ambient, downs2012ambient, alavi2012ambient, rocker2004using}. \textcite{downs2012ambient} define ambient devices as devices that "[...] present dynamic information in an at-a-glance manner and have low attentional requirements".

\section{Well-being: Emotions, Moods, and Sentiments}
A common finding in research regarding remote work is that employees work longer hours, experience more stress, and have difficulties with mental health \autocite{mswellbeing, mulki2009set, qualtricksmental}. A recent study in the context of the global Covid-19 pandemic lists the negative impacts from working from home, such as increased burnout, lack of separation between work and life, and feeling disconnected from co-workers \autocite{mswellbeing}. A Psychological study highlights that the mental health of remote workers should be considered and is very important to be communicated and talked about \autocite{grant2013exploration}. Yet, emotions can get lost or misunderstood inside text messages due to the lack of cues in text-based communication \autocite{hook2008interactional}. For this reason, \textcite{kuwabara2002connectedness} highlights the need for connectedness-oriented communication because it is critical for developing social relationships and harder to do over distance. \textcite{mcduff2012affectaura} further state the usefulness of being able to assess one's emotional state (e.g., when considering mental health issues). Their approach, AffectAura, is developed using different kinds of sensors to predict emotions and provide an overview of them in a diary-like fashion with the purpose of self-reflection \autocite{dullemond2013fixing}. \textcite{guzman2013towards} emphasizes the importance of emotion in software development, however focusing on the emotional state towards a project, not of individuals. MobiMood is a mobile application focusing on individuals by letting them share their moods, but not targetting a work environment \autocite{church2010study}. \textcite{saari2008mobile} developed another mobile application with mood sharing features aimed at knowledge workers. However, while the researchers developed the prototype, their approach's usability and use cases were not studied.


% Apart from the importance of well-being for personal health, there are also more direct impacts on work. For instance, different moods of programmers have been found to have an effect on debugging performance \autocite{khan2011moods}. 

To communicate one's well-being, different types of affective responses exist that can be useful for sharing with the team, namely emotions, moods, and sentiments. Emotions are typical reactions to events and therefore have a definite cause and are typically short-lived. Emotions differ from moods in that moods are longer in duration, have no clear target, and are less intense \autocite{frijda1994varieties, brave2007emotion}. Sentiments can be described as states associated with objects rather than individuals and therefore are relatively permanent \autocite{brave2007emotion}.

When it comes to measuring well-being, the literature does not reach a consensus of one best measurement. However, the valence-arousal dimensional model is most commonly referred to as the better model \autocite{russell1980circumplex, mauss2009measures}. It is a two-dimensional model where the arousal dimension contrasts states of pleasure with states of displeasure, and the arousal dimension contrasts states of low arousal with states of high arousal \autocite{mauss2009measures}. Results of this model can then be used to map onto a discrete set of basic emotions such as surprise, fear, disgust, anger, happiness, or sadness \autocite{brave2007emotion}.