\chapter{Related Work}

...

\section{Remote Work}
Working remotely offers numerous benefits for both the employee and employer compared to traditional co-located work. Benefits on the employee side include a more flexible schedule, higher job productivity, and less time and money spent commuting \cite{flores2019understanding, mulki2009set}. The increased flexibility and autonomy allows employees to more easily deal with their family responsibility and leads to to bigger job satisfaction and higher employee retention \cite{mulki2009set, gajendran2007good, madsen2011benefits}, both very much beneficial for the employer. The employer can further profit from savings in real estate costs and increased productivity \cite{mulki2009set}. In addition to those general benefits, there is another popular reason for building distributed software development teams: the possibility to build teams with talents from all over the world \cite{carmel1999global}. 

However, working remotely creates new challenges for the company and its employees. It is therefore not surprising that much research has been done in this area. We categorize the main challenges of remote work into the following 3 distinct, yet very related and interdependent categories: awareness, communication and well-being. Keeping the challenges of working remotely in mind, we first take a closer look at those three concepts and then take take a look at the existing approaches trying to solve those problems.

It should be noted that the relevance of research in the field is of course elevated due to the Covid-19 pandemic. However, the majority of managers expect to have more flexible work from home policies post-pandemic and employees would like to continue working from home (at least part-time) \cite{msworkindexconnection}, which makes the topic very much relevant also after the pandemic. 

\section{Team Awareness}
Part of the reason for coordination and communication challenges in a remote work environment is the lack of awareness, which is why it is of great interest to us to increase awareness in distributed teams. The term awareness is not really standardized and there exist so many different terms for it. \cite{gutwin1996workspace} defines the most popular framework, defining group awareness as a combination of:

\begin{itemize}
    \item \textit{Informal Awareness} \\
    The general sense of the presence, availability, and activities of others. It is the "glue that facilitates casual interactions" \cite{gutwin1996workspace}. 
    \item \textit{Group-Structural Awareness} \\
    "Group-structural awareness involves the knowledge about people’s
    roles and responsibilities, their positions on an issue, their status, and group processes" \cite{gutwin1996workspace}.
    \item \textit{Social Awareness} \\
    "Social awareness is the information that a person maintains about others in a social or conversational context"\cite{gutwin1996workspace}. It includes things as the attention state of the other person, their emotions, or the level of interest \cite{gutwin1996workspace}, or whether a person can be disturbed \cite{gutwin1995support}.
    \item \textit{Workplace Awareness} \\
    Defines the awareness that results from the 'real-time' combination of elements workers keep track of when working together. Such elements could be people, actions, objects, and many more \cite{gutwin1995support}. 
\end{itemize}

The term team awareness used by us is closely related to the above defined workplace awareness. It is important to note that those four awareness types are not excluding but rather overlapping with each other. Put differently, informal, social, and group-structural awareness are all part of workplace awareness. 

All of those awareness types are of relevance for software developers. A study showing that developers checked the availability status of their co-workers almost as many times as their compiler output \cite{ko2007information} indicates the importance of informal awareness. Providing group-structural is important because of difficulties when trying to find experts in a distributed team \cite{herbsleb2003empirical}. Social awareness is highly relevant due to the high communication needs of software developers \cite{perry1994people}. Additionally, with less face-to-face communication and more computer-mediated communication it is consequently more difficult to transfer emotional information \cite{rivera1996effects}. Last but not least, workplace awareness is crucial for collaboration and coordination inside software development teams.

\section{Team Communication}

%Modularisation of source code is a very popular design principle in the context of software engineering. However, software development is considered to be a highly complex task, meaning that such that even in very well designed systems with high modularisation, some dependencies remain \cite{cataldo2007coordination}. Many studies have shown that successful coordination during the process of software development leads to higher performance \cite{kraut1995coordination}. Distributed teams have a disadvantage coordinating compared to traditional collocated teams \cite{herbsleb2003empirical}.

Research states that coworkers are the most used source of information used by developers \cite{ko2007information}, emphasising the importance of team communication inside software development teams. In the context of remote work, studies find different results regarding the communication frequency: while \cite{kraut1988patterns, allen1984managing} finds a decrease in communication, \cite{mulki2009set} find increased communication. A possible reason for more communication include the need of remote workers to over-communicate their availability status to their co-workers \cite{koehne2012remote}. Regardless of communication frequency, working remotely and thus using software to communicate leads to have more misunderstandings due to missing cues and thus reduces communication effectiveness \cite{mulki2009set}. Other researchers argue that the reason for those misunderstandings is the fact text-based communication (which is often used in software development) has very limited capacity and thus a lot of socio-emotional information (non-verbal cues) is missing \cite{hassib2017heartchat}. Face-to-face communication is still very important for many developers \cite{storey2016social}. The lack thereof, which is caused by working remotely, can lead to workplace isolation and makes it harder to develop personal relationships and build trust \cite{mulki2009set}. \cite{gajendran2007good} does state that working from home with high-intensity (more than 2.5 days a week) harmed relationships with coworkers, something that is enforced because of Covid-19.
Other consequences of remote work are that software developers have difficulties getting help when working remotely and finding experts \cite{herbsleb2003empirical, espinosa2007team}, which is a significant driver for team performance \cite{faraj2000coordinating}. Working remotely reduced communication ease by requiring active effort to bring back ad-hoc meetings \cite{miller2021your}. This is likely more pronounced because of the global pandemic which forced teams to switch from collocated to remote work, making communication as easy as possible should be desirable regardless. 
\\\\
\textbf{Informal Communication}\\
\cite{kraut1990informal} defines informal communication as "[...] communication that is spontaneous, interactive and rich". The main difference to formal communication is the lack of planning, the content of the communication is not known in advance. They further state that over 85\% of all conversations were of informal nature and that informal communication happens more often if there is short physical distance between parties. Similarly, \cite{hinds2005understanding} find that members of distributed teams engage less in informal communications. Generally, informal communication is crucial for achieving high productivity and social goals \cite{kraut1990informal}. More concretely, in the field of software development informal communication plays a critical role due to the speed at which informal communication distributes knowledge \cite{french1998study, mockus2001challenges}. Also, informal communication can increase awareness, enabling developers to work efficiently \cite{herbsleb2001global}. In the ever-changing field of agile software development this is particularly useful because requirements can change and formal communication channels cannot spread the news as fast. In addition, informal communication has been shown to be important for conflict identification and handling \cite{hinds2005understanding}. Since teams with a high degree of social interactions often have better team cohesion \cite{staehle2014management}, and informal communication is normally much more frequent than formal forms of communication \cite{kraut1990informal}, increasing informal communication is likely highly beneficial for team cohesion.
\\\\
\textbf{Interruptions and Work Fragmentation}\\
Interruptions at the work place have been shown to decrease productivity because when being interrupted users require more time to complete tasks, make more errors and experience more annoyance \cite{bailey2006need}. Due to the lack of attentional states available when working remotely, interruptions cannot be timed as carefully \cite{mark2005no}.

\section{Well-being} 
Common findings in research regarding remote work is that employees work longer hours, more stress and difficulties \cite{mswellbeing, mulki2009set, qualtricksmental}. A recent study in the context of the global Covid-19 pandemic lists the negative impacts from working from home such as increased burnout, lack of separation between work and life, and feeling disconnected from co-workers \cite{mswellbeing}. A Psychological study highlights that the mental health of remote workers should be considered and is very important to be communicated and talked about \cite{grant2013exploration}. For those reasons, we see a need for better and easier communication of well-being and mental health. In Ambient Teams we achieve this by offering the possibility to share emotions.
\\\\
\textbf{Emotions} \\
Not only are emotions important for well-being, but they also have more direct influence on work. For instance, different moods of programmers have been found to have an impact on debugging performance \cite{khan2011moods}. \cite{mcduff2012affectaura} further states the usefulness of being able to assess one's emotional state (e.g. when considering mental health issues).

When it comes to measuring emotions, the literature does not really reach consensus of one best measurement, however the valence-arousal dimensional model is most commonly referred to as the ``better'' model \cite{russell1980circumplex, mauss2009measures}. 

\section{Existing Approaches}

\textbf{Increasing Awareness} \\
\cite{rocker2012informal} emphasizes the lack of information present in current communication media such as video conferencing, phone calls, or email, which are essential for awareness and thus informal communication. However, there exist many tools built on top of those media to achieve just that. In fact, a wealth of research developed approaches to increase awareness in distributed teams. Popular tools specifically made for software development teams focus on providing awareness by on work items, developers' activities (e.g. which files they have opened or recently changed) and thus put the code base and tasks in the foreground of coordination \cite{biehl2007fastdash, jakobsen2009wipdash, eick1992seesoft, deline2005easing}. While the team awareness gained by those tools allows developers to understand who you are working with and what they are working on, and what the impact of a change can have on others, which is essential for successful collaboration \cite{dourish1992awareness}, they only cover a very limited view of awareness.

While the majority of the above mentioned awareness tools require user interactions to be useful, there have also been attempts for creating ambient tools to raise awareness in the work environment \cite{morrison2020facilitating, otjacques2006ambient, downs2012ambient, alavi2012ambient}. Unfortunately, all of them focus on a very specific industry, and while they increase informal awareness, none of them focus on social awareness, both of which are types of awareness important at the work place \cite{greenberg1996awareness}.

The focus of Ambient Teams is to raise team/group awareness in an ambient manner, but by putting teams and people in the foreground, similar to \cite{whittaker2004contactmap}, yet with the goal of fostering informal communication instead of email. 
\\\\
\textbf{Fostering (informal) Communication} \\
There are numerous approaches fostering informal communication inside distributed teams. One of the earliest proposed solutions for fostering informal communication in distributed teams was VideoWindow \cite{fish1990videowindow}. Despite being an early solution, they already identified two important key requirements such a system has to offer, namely low personal cost and the need for a visual channel: if the costs for initiating a conversations are too high, the system will not be useful, and, the visual channel plays an important role in recognizing the presence of other people. \cite{sasaki1999video} developed a 'hallway' system which was able to raise awareness and helped to indicate that one might have a question, but failed to promote casual interactions. In comparison, \cite{lou2012presencescape} manage to provision awareness information that is relevant to engage in casual conversations and a low-effort mechanism to initiate such informal conversations. It does so by providing social cues which are useful for understanding the availability of others and thus creating a context for subsequent communication. Micro-blogging is another strategy that allowed software engineers to share activities and moods with other team members with the result of feeling more connected to each other \cite{dullemond2013fixing}. More recent projects, being released as a consequence of the global pandemic (Branch\footnote{\url{https://branch.gg}}, Reslash\footnote{\url{https://reslash.co}}, Wonder\footnote{\url{https://wonder.me}}), follow a similar goal set of increasing informal/spontaneous communication, however with the same downside: requiring a fair bit of user interaction. Additionally, the possibility to move around and form differnet groups in real-time might not really add that much value in a work environment where teams are already built. Thus, Ambient Teams avoids this overhead and its goal is to be as ambient as possible, while still increasing social awareness and fostering informal communication. When fostering communication, it becomes important to not foster such communication in inopportune moments. Since concepts like presence/availability awareness, yet another part of informal awareness, are useful in reducing the number of interruptions \cite{mark2005no}, Ambient Teams allows users to communicate their availability/interruptibility. 

Additionally, by increasing informal communication we want to improve well-being, another major challenge in working remotely in context of the current pandemic.
\\\\
\textbf{Monitoring/Recognizing Emotions} \\
AffectAura is an approached developed using different kinds of sensors to predict emotions and provide an overview of them in a diary-like fashion \cite{dullemond2013fixing}. While its purpose is mainly self-reflection, we expect value in the ability to easily share emotional state with co-workers, serving as both self-reflection and information (awareness) provider for others. \cite{guzman2013towards} emphasises the importance of emotion in software development, however focusing on the emotional state towards a project, not of individuals. 