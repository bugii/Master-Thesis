\chapter{Related Work}
Working remotely offers numerous benefits for both the employee and employer compared to traditional co-located work. Benefits on the employee side include a more flexible schedule, higher job productivity, and less time and money spent commuting \autocite{flores2019understanding, mulki2009set}. The increased flexibility and autonomy allows employees to more easily deal with their family responsibility and leads to to bigger job satisfaction and higher employee retention \autocite{mulki2009set, gajendran2007good, madsen2011benefits}, both very much beneficial for the employer. The employer can further profit from savings in real estate costs and increased productivity \autocite{mulki2009set}. In addition to those general benefits, there is another popular reason for building distributed software development teams: the possibility to build teams with talents from all over the world \autocite{carmel1999global}.

However, working remotely creates new challenges for the company and its employees. It is therefore not surprising that much research has been done in this area, most of which coming from research in Computer Supported Collaborative Work (CSCW). The general goal is to support distributed teams in accomplishing work as effective and efficient as possible. We identified four main challanges that are the result of working remotely. All of them, together with and existing solutions, are discussed in this chapter.

% We categorize the main challenges of remote work into the following 3 distinct, yet very related and interdependent categories: awareness, communication and well-being. Keeping the challenges of working remotely in mind, we first take a closer look at those three concepts and then take take a look at the existing approaches trying to solve those problems.

% \textbf{Interruptions and Work Fragmentation}\\
% Interruptions at the work place have been shown to decrease productivity because when being interrupted users require more time to complete tasks, make more errors and experience more annoyance \autocite{bailey2006need}. Due to the lack of attentional states available when working remotely, interruptions cannot be timed as carefully \autocite{mark2005no}.

\section{Workplace Isolation}
\textcite{marshall2007workplace} defines workplace isolation as the "[...] desire to be part of the network of colleagues who provide help and support in specific work-related needs. It represents employees’ perceptions of availability of co-workers, peers, and supervisors for work-based social support." They further suggest a categorisation into social isolation and organizational isolation. Oranzisational isolation comes from the perception that remote workers might feel "out of sight, out of mind" \autocite{bailey1999advantages}. Social isolation relates to the fact that remote workers miss informal chats, spontaneous discussions, and meetings around the water cooler \autocite{cooper2002telecommuting}. For those reasons, a closer look at informal communication will be given in the following.

\section{Communication}
%Modularisation of source code is a very popular design principle in the context of software engineering. However, software development is considered to be a highly complex task, meaning that such that even in very well designed systems with high modularisation, some dependencies remain \autocite{cataldo2007coordination}. Many studies have shown that successful coordination during the process of software development leads to higher performance \autocite{kraut1995coordination}. Distributed teams have a disadvantage coordinating compared to traditional collocated teams \autocite{herbsleb2003empirical}.
Research states that coworkers are the most used source of information used by developers \autocite{ko2007information}, emphasising the importance of team communication inside software development teams. In the context of remote work, studies find different results regarding the communication frequency: while \textcite{kraut1988patterns, allen1984managing} find a decrease in communication, \textcite{mulki2009set} find increased communication. A possible reason for more communication include the need of remote workers to over-communicate their availability status to their co-workers \autocite{koehne2012remote}. Reasons for communication reduction could be the active effort to bring back ad-hoc meetings \autocite{miller2021your}. Regardless of communication frequency, working remotely and thus using software to communicate leads to have more misunderstandings due to missing cues and thus reduces communication effectiveness \autocite{mulki2009set}. Other researchers argue that the reason for those misunderstandings is the fact text-based communication (which is often used in software development) has very limited capacity and thus a lot of socio-emotional information (non-verbal cues) is missing \autocite{hassib2017heartchat}. Face-to-face communication is still very important for many developers \autocite{storey2016social} and a lack thereof, which is caused by working remotely, can lead to workplace isolation making it harder to develop personal relationships and build trust \autocite{mulki2009set}. \textcite{gajendran2007good} state that working from home with high-intensity (more than 2.5 days a week) harmed relationships with coworkers, something that is enforced because of Covid-19.


% This is likely more pronounced because of the global pandemic which forced teams to switch from collocated to remote work, making communication as easy as possible should be desirable regardless.


\subsection{Informal Communication}

To faciliate communication, some awareness tools also have basic informal communication tools built into them. \textcite{cheng2003jazzing} introduces JazzBand, a visualition of the team members with the goal of increasing peripheral awareness enhanced with status messages and chat functionality. By being a IDE Plugin however, we argue the communication likely is tightly coupled to the code, thus being work-related and only used when actually coding, and limited to software developers. AmbientTeams expands the JazzBand ideas and bundles them in a more general-purpose awareness desktop application with informal communication functionality.

\textcite{kraut1990informal} define informal communication as "[...] communication that is spontaneous, interactive and rich". Differences to formal communication include lack of planning and the fact the content of the communication is not known in advance. They further state that over 85\% of all conversations are of informal nature and that informal communication happens more often if there is short physical distance between parties. Similarly, \textcite{hinds2005understanding} find that members of distributed teams engage less in informal communications. This is unfortunate since informal communication is crucial for achieving high productivity and social goals \autocite{kraut1990informal}. More concretely, in the field of software development informal communication plays a critical role due to the fast speed at which informal communication distributes knowledge across a team or company \autocite{french1998study, mockus2001challenges}. Also, informal communication can increase awareness, enabling developers to work efficiently \autocite{herbsleb2001global}. In the ever-changing field of agile software development this is particularly useful because requirements can change and formal communication channels cannot spread the news as fast. In addition, informal communication has been shown to be important for conflict identification and handling \autocite{hinds2005understanding}. Since teams with a high degree of social interactions often have better team cohesion \autocite{staehle2014management}, and informal communication is normally much more frequent than formal forms of communication \autocite{kraut1990informal}, increasing informal communication is likely highly beneficial for team cohesion.

Because of those benefits, it is no surprise that there are numerous approaches fostering informal communication inside distributed teams. One of the earliest proposed solutions for fostering informal communication in distributed teams was VideoWindow \autocite{fish1990videowindow}. Despite being an early solution, they already identified two important key requirements such a system has to offer, namely low personal cost and the need for a visual channel: if the costs for initiating a conversations are too high, the system will not be useful, and, the visual channel plays an important role in recognizing the presence of other people. \textcite{sasaki1999video} developed a 'hallway' system which was able to raise awareness and helped to indicate that one might have a question, but failed to promote casual interactions. In comparison, \textcite{lou2012presencescape} manage to provision awareness information that is relevant to engage in casual conversations and a low-effort mechanism to initiate such informal conversations. It does so by providing social cues which are useful for understanding the availability of others and thus creating a context for subsequent communication.

As a consequence of the global pandemic, many commercial tools have been published recently. Branch\footnote{\url{https://branch.gg}}, Reslash\footnote{\url{https://reslash.co}}, Wonder\footnote{\url{https://wonder.me}}, or Gather\footnote{\url{https://gather.town}} follow a similar goal set of increasing spontaneous, informal communication. By mimicing the real office, those virtual office approaches all come with the same downside: requiring a fair bit of user interaction due to the visually complex interface. We argue that this adds a lot of unnecessary overhead and reduces long term usability. Thus, AmbientTeams avoids this overhead and its goal is to be as ambient as possible, while still fostering informal communication. Tandem\footnote{\url{https://tandem.chat/}} takes a similar approach in that it is less playful than the other commercial tools, but still very focused on collaboration and communication. In comparison, AmbientTeams additionally includes a glancable overlay and team members' moods and status messages.


\section{Awareness}
Part of the reason for coordination and communication challenges in a remote work environment is the lack of awareness, which is why it is of great interest to us to increase awareness in distributed teams. \textcite{gutwin1996workspace} defines group awareness as a combination of:

\begin{itemize}[itemsep=0ex, parsep=0ex, leftmargin=*]
      \item \textbf{Informal Awareness} \\
            The general sense of the presence, availability, and activities of others. It is the "glue that facilitates casual interactions" \autocite{gutwin1996workspace}.
      \item \textbf{Group-Structural Awareness} \\
            "Group-structural awareness involves the knowledge about people’s
            roles and responsibilities, their positions on an issue, their status, and group processes" \autocite{gutwin1996workspace}.
      \item \textbf{Social Awareness} \\
            "Social awareness is the information that a person maintains about others in a social or conversational context"\autocite{gutwin1996workspace}. It includes things as the attention state of the other person, their emotions, or the level of interest \autocite{gutwin1996workspace}, or whether a person can be disturbed \autocite{gutwin1995support}.
      \item \textbf{Workplace Awareness} \\
            Defines the awareness that results from the 'real-time' combination of elements workers keep track of when working together. Such elements could be people, actions, objects, and many more \autocite{gutwin1995support}.
\end{itemize}

It is important to note that those four awareness types are not excluding but rather overlapping with each other. Put differently, informal, social, and group-structural awareness are all part of workplace awareness. In the case of software developers for instance, a study shows that developers checked the availability status of their co-workers almost as many times as their compiler output \autocite{ko2007information}. This indicates the importance of informal awareness. Providing group-structural is important because of difficulties when trying to find experts in a distributed team \autocite{herbsleb2003empirical}. Social awareness is highly relevant due to the high communication needs of software developers \autocite{perry1994people}. Additionally, with less face-to-face communication and more computer-mediated communication it is consequently more difficult to transfer emotional information \autocite{rivera1996effects}.

Despite the seamingly clear categorisation of awareness above, the literature does not agree on one common categoristaion and terms such as general awareness, peripheral awareness, co-existent and cooperation awareness, and objective self-awareness are used instead. Due to the popularity and granularity of the above definition, it is the definition of choice for this work.

To address the problem of missing awareness when working remotely, a wealth of research developed approaches to increase awareness in distributed teams. Popular tools specifically made for software development teams focus on providing awareness by on work items, developers' activities (e.g. which files they have opened or recently changed) and thus put the code base and tasks in the foreground of coordination \autocite{biehl2007fastdash, jakobsen2009wipdash, eick1992seesoft, deline2005easing}. While the team awareness gained by those tools allows developers to understand who you are working with and what they are working on, and what the impact of a change can have on others, which is essential for successful collaboration \autocite{dourish1992awareness}, they only cover a very limited view of awareness by providing very limited social or emotional information. \textcite{garcia1999emotional} already mentioned the need for emotional awareness inside groupware in 1999. Despite yet there is very little research done in that field to date. Similar to \autocite{mora2011supporting}, we argue the awareness of moods in a work environment is underrepresented in research, especially in a society where many are facing mental challenges caused by the global COVID-19 pandemic.

While the majority of the above mentioned awareness tools require user interactions to be useful, there have also been attempts for creating ambient approaches to raise awareness in the work environment \autocite{morrison2020facilitating, otjacques2006ambient, downs2012ambient, alavi2012ambient, rocker2004using}. \textcite{downs2012ambient} define ambient devices as devices that "[...] present dynamic information in an at-aglance manner and have low attentional requirements". Unfortunately, while they increase informal awareness in a non-intrusive way, none of them focus on social awareness, both of which are types of awareness important at the work place \autocite{greenberg1996awareness}. Additionally, many of the above approach include physical devices (e.g. from \autocite{ downs2012ambient, alavi2012ambient, rocker2004using})which might not be suitable for a remote team setting due to size of the device or the device with the awareness information would simply not be visible to remote team members.

The focus of AmbientTeams is to raise team/group awareness in an ambient manner, but by putting teams and people in the foreground, similar to \autocite{whittaker2004contactmap}, yet with the goal of fostering informal communication instead of email.

\section{Well-being: Emotions, Moods, and Sentiments}
A Common finding in research regarding remote work is that employees work longer hours, experience more stress, and have difficulties with mental health \autocite{mswellbeing, mulki2009set, qualtricksmental}. A recent study in the context of the global Covid-19 pandemic lists the negative impacts from working from home such as increased burnout, lack of separation between work and life, and feeling disconnected from co-workers \autocite{mswellbeing}. A Psychological study highlights that the mental health of remote workers should be considered and is very important to be communicated and talked about \autocite{grant2013exploration}. For those reasons, we see a need for better and easier communication of well-being and mental health and tools that increase the feeling of connectedness among teams. This is a similar finding to \textcite{kuwabara2002connectedness} who highlights the need for connectedness-oriented communication because it is ciritical for developing social relationships and harder to do over distance. \textcite{mcduff2012affectaura} further state the usefulness of being able to assess one's emotional state (e.g. when considering mental health issues). Their approach, AffectAura, is developed using different kinds of sensors to predict emotions and provide an overview of them in a diary-like fashion with the purpose of self-reflection \autocite{dullemond2013fixing}. \textcite{guzman2013towards} emphasises the importance of emotion in software development, however focusing on the emotional state towards a project, not of individuals. MobiMood is a mobile application focusing on individuals by letting them share their moods, but not targetting a work environment \autocite{church2010study}. \textcite{saari2008mobile} developed another mobile application with mood sharing features aimed at knowledge workers. However, while the researchers developed the prototype, the usability and use cases for their approach were not studied. Both of which will be analysed in the scope of this study. Further, a mobile application might not be ideal to use in a work environment since this might also introduce more room for interruptions.

% Apart from the importance of well-being for personal health, there are also more direct impacts on work. For instance, different moods of programmers have been found to have an impact on debugging performance \autocite{khan2011moods}. 

In order to communicate one's well-being, there are different types of affective responses that could useful for sharing with the team, namely emotions, moods, and sentiments. Emotions are typically reactions to events and therefore have a concrete cause and are typically short-lived. Emotions differ from moods in that moods are longer in duration, have no clear target and are less intense \autocite{frijda1994varieties, brave2007emotion}. Sentiments can be described as states associated with objects rather than individuals and therefore are of a rather permanent nature \autocite{brave2007emotion}.

When it comes to measuring affect, the literature does not really reach consensus of one best measurement, however the valence-arousal dimensional model is most commonly referred to as the ``better'' model \autocite{russell1980circumplex, mauss2009measures}. It is a two-dimensional model where the arousal dimension contrasts states of pleasure with states of displeasure, and the arousal dimension contrasts states of low arousal with states of high arousal \autocite{mauss2009measures}. Results of this model can then be used to map onto a discrete set of basic emotions such as surprise, fear, disgust, anger, happiness, or sandness \autocite{brave2007emotion}.